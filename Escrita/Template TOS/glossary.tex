\newacronym{HEVC}{HEVC}{High Efficiency Video Coding}
\newacronym{TV}{TV}{Television}
\newacronym{CMOS}{CMOS}{Complementary metal–oxide–semiconductor}
\newacronym{UHD}{UHD}{Ultra-High-Definition}
\newacronym{CODEC}{Codec}{Encoder-Decoder}
\newacronym{fps}{fps}{frames per second}
\newacronym[plural=Cathode Ray Televisions (CRTs)]{CRT}{CRT}{Cathode Ray Television}
\newacronym{ic}{IC}{Integrated Circuit}
\newacronym{gpu}{GPU}{Graphical Processing Unit}
\newacronym{av1}{AV1}{AOM Video 1}
\newacronym{aomedia}{AOM}{Alliance for Open Media}
\newacronym{mpeg}{MPEG}{Motion Picture Experts Group}
\newacronym{fft}{FFT}{Fast Fourier Transform}
\newacronym{dft}{DFT}{Discrete Fourier Transform}
\newacronym{dct}{DCT}{Discrete Cosine Transform}
\newacronym{wht}{WHT}{Walsh-Hadamard Transform}
\newacronym{wgn}{WGN}{White Gaussian Noise}
%\newacronym{jpeg}{JPEG}{Joint Photographic Experts Group}

\newglossaryentry{rgb}{%
  name={RGB},
  description={Color space based on the addition of Red, Green and Blue components for complex color representation}
}

\newglossaryentry{codec}{
  name={Codec},
  description={Encoder-Decoder. Also referred to the method of compressing and decompressing a video sequence}
}

\newglossaryentry{interlacing}{
  name={Interlaced scanning},
  description={Technique used by televisions for broadcasting and displaying, where only odd or even numbered lines of a frame are transmitted/displayed at a time, alternately}
}

\newglossaryentry{progressive}{
  name={Progressive scanning},
  description={Technique used by more recent screens, where each frame is displayed as a whole, from top to bottom, and left to right}
}

\newglossaryentry{VP89}{
  name={VP8/VP9},
  description={Open-format video codecs developed by Google, released in 2008 and 2013, respectively}
}

\newglossaryentry{libaom}{
  name={libaom},
  description={Reference software for AV1, released by Google in June 2018}
}

\newglossaryentry{JPEG}{
  name={JPEG},
  description={Still image compression format, developed by the Joint Photographic Experts Group (JPEG)}
}

\newglossaryentry{pixel}{
  name={Pixel},
  description={Picture Element}
}
\glsaddall
