\cleardoublepage
\chapter{Developed Architectures}

%%%%%%%%%%%%%%%%%%%%%%%%%%%%%%%%%%%%%%%%%%%%%%%%%%%%%%%%%%%%%%%%%%%%%%%%%%%%%%
\section{Objectives and Workflow}

The previous chapter presented some characteristics of the current state of \emph{libaom}'s \emph{Transform} stage which might compromise its performance, the most relevant being the unnecessary flexibility in the representation of cosine approximations.

In order to undertake these opportunities, and improve the overall encoder performance, new architectures for the studied stage were developed.

The developed implementations tackled the forward \emph{DCT}, since it was the \emph{kernel} that would have the most impact on encoder performance. As the \emph{IDCT} is shared between encoder and decoder, and due to the added complexity, no changes were done to this block, as it acts with accordance with the established standard, as mentioned previously. 

%%%%%%%%%%%%%%%%%%%%%%%%%%%%%%%%%%%%%%%%%%%%%%%%%%%%%%%%%%%%%%%%%%%%%%%%%%%%%%
\section{Matrix Multiplication Implementation}

The first test was the application of the simplest integer \emph{DCT}, done by the multiplication of the input vector by a scaled up version of the transform matrix, $\mathbf{F}$, firstly shown in Equation \ref{eq:DCT2}. 

The original integer transform matrix is shown in equation \ref{eq:matscale}.

\begin{equation} \label{eq:matscale}
    \begin{gathered}
        \mathbf{F}_{x,u} = \beta(u)\cos\left(\frac{(2x+1)u\pi }{2L}\right)\,0\leq u,x < L \\
        \Downarrow \\
        \mathbf{F} = \sqrt{\frac{2}{L}}  \begin{bmatrix}
            \sqrt{\frac{1}{2}}                                  & \sqrt{\frac{1}{2}}                                & \dots & \sqrt{\frac{1}{2}} \\
            \cos\left(\frac{\pi}{2L}\right)    & \cos\left(\frac{3\pi}{2L}\right) & \dots & \cos\left(\frac{(2(L-1)+1)\pi}{2L}\right) \\
            \vdots     & \vdots     & \ddots & \vdots       \\
            \cos\left(\frac{(L-1)\pi}{2L}\right)    & \cos\left(\frac{3(L-1)\pi}{2L}\right) & \dots & \cos\left(\frac{(2(L-1)+1)(L-1)\pi}{2L}\right) \\
        \end{bmatrix} 
    \end{gathered}
\end{equation}

As mentioned previously, the floating point coefficients bring a number of disadvantages on an hardware implementation, from increased calculation overheads, to encoder/decoder mismatches. 

In order to address these problems, a scale and rounding operation was performed, as shown in Equation \ref{eq:matscale}, where $K$ represents the number of bits of the scaled coefficients.

\begin{equation} \label{eq:matscale}
    \nint*{\mathbf{F}_K}   = \nint*{2^K \mathbf{F}}
\end{equation}

However, due to the rectangular block sizes allowed in \emph{AV1}, the factor $\sqrt{\frac{2}{L}}$ isn't considered in the kernels themselves. Instead, the transformed outputs get scaled at a later stage. This way, the implemented transform matrix is

\begin{equation} \label{eq:matscale2}
    \begin{gathered}
        \nint*{\mathbf{F}_K}   = \nint*{2^K \sqrt{\frac{L}{2}}\mathbf{F}} \\
        = \nint*{2^K \begin{bmatrix}
                        \sqrt{\frac{1}{2}}                                  & \sqrt{\frac{1}{2}}                                & \dots & \sqrt{\frac{1}{2}} \\
                        \cos\left(\frac{\pi}{2L}\right)    & \cos\left(\frac{3\pi}{2L}\right) & \dots & \cos\left(\frac{(2(L-1)+1)\pi}{2L}\right) \\
                        \vdots     & \vdots     & \ddots & \vdots       \\
                        \cos\left(\frac{(L-1)\pi}{2L}\right)    & \cos\left(\frac{3(L-1)\pi}{2L}\right) & \dots & \cos\left(\frac{(2(L-1)+1)(L-1)\pi}{2L}\right) \\
                    \end{bmatrix} 
                }
    \end{gathered}
\end{equation}

This way, the  transformed outputs are calculated through

\begin{equation}
    \mathcal{G}(u) = \left(\nint*{\mathbf{F}_K} g(x)\right)>>K, \, 0 < u,x < L
\end{equation}

For an $L$ length vector, the calculation of the transformed vector implies $L^2$ additions and $L^2$ multiplications, which leads to the main disadvantage of such implementation. For larger vectors, this operation becomes too demanding in terms of memory and complexity.