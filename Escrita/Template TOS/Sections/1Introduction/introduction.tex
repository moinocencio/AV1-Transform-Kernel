\cleardoublepage
\pagenumbering{arabic}
\chapter{Introduction}

\section{\textcolor{red}{Background and Motivation}}
\todo[inline,color=green]{* Desenvolvimento da qualidade de vídeo ao consumidor final}
\todo[inline,color=red]{* Depêndencia das baterias para aplicações móveis}
\todo[inline,color=red]{* Hardware dedicado melhora performance e consumo energético}

Since the spark of television research in 1887, a tremendous investment has been put into increasing the quality of images, cameras and screens that display them \cite{schubinWhatSparkedVideo2017}.

In the early years of mechanical television, this desire was pursued by making changes to the \textit{Nipkow} disks, up to the decline of the mechanical TV, around the 1930's. The consequential rise of all-electronic TVs started with the capture of images with the same cathode tubes put into the televisions, with broadcasts of the live analog recordings, since there were no available methods of storing images, up to 1955, with the development of the open-reel magnetic tape \cite{jacobsBriefHistoryVideo}.

The evolution of \Gls{CMOS} technologies however, led to the downfall of camera tubes, and to the rise of image capture to a digital sensor, that allowed better image captures and lower demands in terms of storage space. However, with the desire for higher fidelity video, the quantity of information captured also increased. Whether by increasing the sensor resolution, color bit depth or frame rate, the captured video sequences have increased its size throughout the years. For instance, for a video of $640 \times 360$ (considered as a low resolution), at 30 \gls{fps}, considering each captured color (\GLS{rgb}
\todo[size=\tiny]{Falar dos color spaces mais à frente?}
) is represented with 8 bits, there is approximately 166 Million bits per second (Mbps) of captured information. This means that a short 5 minute video would occupy more than 6 Giga Bytes (GB) of memory. This aspect gets more severe once higher resolutions are considered. For newer standards such as 4K \Gls{UHD} ($3840 \times 2160$) or 8K UHD ($ 7680 \times 4320$), under the same conditions, a ten minute video would occupy 448 GB and 1792 GB of raw data, respectively.

\todo[inline]{Insert here CISCO forecast}

This problem has led to the introduction of a new concept: \textit{Video Compression} \footnote{Also called \textit{Video Coding}.}, which is the process of reducing the size of a video sequence, while still maintaining its playback capabilities. The \Gls{codec} takes advantage of redundant information present on the raw data to reduce the size of the video, without heavily modifying the original picture or its quality
\todo[size=\tiny]{Referência para secção mais à frente?}
. 

The first form of video compression, \gls{interlacing}, dates from 1940, and was purely analog. This solution was introduced with the intent of reducing the necessary broadcasting bandwidth for old \glspl{CRT}, without decreasing the displayed fps. And even though this technique has been implemented over more than seventy years, it has proven to be so efficient that most TV channels today still use interlaced broadcasting.

However, analog television is now obsolete, as well as CRTs. The massive developments in \Gls{ic} fabrication led to the rise of the digital era we now live in. Therefore, most screens (be it televisions, monitors or cellphones) use digital, \gls{progressive}. Therefore, the use of analog compression techniques wasn't applicable. Accordingly, the evolution of digital video led to the development of digital compression techniques, such as the one presented in this text.


\printbibliography[heading=subbibliography]
\addcontentsline{toc}{section}{References}
