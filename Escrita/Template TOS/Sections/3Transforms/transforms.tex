\cleardoublepage
\chapter{Video Coding Transforms}

%%%%%%%%%%%%%%%%%%%%%%%%%%%%%%%%%%%%%%%%%%%%%%%%%%%%%%%%%%%%%%%%%%%%%%%%%%%%%%
\section{\textcolor{red}{Introduction}}

\textcolor{red}{As mentioned previously,} the basic principle behind the compression of video, is the reduction of inter-pixel/inter-symbol correlation. The various integral blocks of a video compression system try to accomplish this objective through different strategies. The \emph{Intra-frame} and \emph{Inter-frame Prediction} exploit spatial and temporal correlation, respectively. With the subtraction of the input by the output of one of these blocks, and the attainment of the \emph{residue}, the next compression stage is made in the \emph{Transform} block
\todo{as seen in ...}
,which is the focus of this work.

\todo[inline,color=red!40]{Verify accordance with previous chapter}

The technique implemented by this process relies on the energy compaction in the frequency domain to reduce the correlation within a frame block, i.e. the input of the Transform block is evaluated on its main frequencies --- the \emph{transform coefficients} --- on a spatial and/or temporal domain, similarly to the process executed on an \gls{fft}. Once each block is quantized on these coefficients, the compression is made with the removal of the least significant ones, on the \emph{Quantization} stage. The intent of the \emph{transform} is to split the image into a set of predefined coefficients.

A useful interpretation of this process is to see it as the decomposition of the input as a set of basis vectors (1D transforms) or images/matrices (2D transforms). The transformation outputs , $y_i$, can be seen as the weights of each basis vector/image, $\vec{e_i}$, that summed return the input, $\vec{z}$, i.e.

\begin{equation}
    \vec{z} = \sum_{i=1}^{N} y_i \vec{e_i}
\end{equation}
which means that the coefficients are related to the amount of correlation between the input and each basis component, and can be obtained with the \emph{inner product} of the input and each basis vector %\cite[sec. 4.1.4 \& 4.2.2]{shiImageVideoCompression2008}.
.

\begin{equation} \label{eq:coef_vec}
    y_i = \vec{e_i}^T \vec{z}
\end{equation}

Considering a 2D image, $g(x,y)$, and its corresponding transformed coefficients, $T(u,v)$, where $(x,y)$ are the pixel coordinates, and $(u,v)$ are the corresponding coordinates in the transform domain, we can obtain an analogous version of equation \ref{eq:coef_vec} as

\begin{equation}
    T(u,v) = \sum_{x=0}^{N-1}\sum_{y=0}^{M-1}g(x,y)f(x,y,u,v)
\end{equation}

Similarly, we can re-obtain the original picture

\begin{equation}
    g(x,y) = \sum_{u=0}^{N-1}\sum_{v=0}^{M-1}T(u,v)i(x,y,u,v)
\end{equation}
where $f(x,y,u,v)$ and $i(x,y,u,v)$ are the \emph{forward} and \emph{inverse transformation kernels} \cite[ch. 4]{shiImageVideoCompression2008}.


%%%%%%%%%%%%%%%%%%%%%%%%%%%%%%%%%%%%%%%%%%%%%%%%%%%%%%%%%%%%%%%%%%%%%%%%%%%%%%
\section{\textcolor{red}{Background}}

%%%%%%%%%%%%%%%%%%%%%%%%%%%%%%%%%%%%%%%%%%%%%%%%%%%%%%%%%%%%%%%%%%%%%%%%%%%%%%
\section{\textcolor{red}{Used Transformation Kernels}}

\printbibliography[heading=subbibliography]
\addcontentsline{toc}{section}{References}