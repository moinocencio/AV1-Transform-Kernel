\newacronym{HEVC}{HEVC}{High Efficiency Video Coding}
\newacronym{TV}{TV}{Television}
\newacronym{CMOS}{CMOS}{Complementary metal–oxide–semiconductor}
\newacronym{UHD}{UHD}{Ultra-High-Definition}
\newacronym{CODEC}{Codec}{Encoder-Decoder}
\newacronym{fps}{fps}{Frames per Second}
\newacronym[plural=Cathode Ray Televisions (CRTs)]{CRT}{CRT}{Cathode Ray Television}
\newacronym{ic}{IC}{Integrated Circuit}
\newacronym{gpu}{GPU}{Graphical Processing Unit}
\newacronym{av1}{AV1}{AOM Video 1}
\newacronym{aomedia}{AOM}{Alliance for Open Media}
\newacronym{mpeg}{MPEG}{Motion Picture Experts Group}
\newacronym{fft}{FFT}{Fast Fourier Transform}
\newacronym{dft}{DFT}{Discrete Fourier Transform}
\newacronym{dct}{DCT}{Discrete Cosine Transform}
\newacronym{adst}{ADST}{Asymetric Discrete Sine Transform}
\newacronym{wht}{WHT}{Walsh-Hadamard Transform}
\newacronym{wgn}{WGN}{White Gaussian Noise}
\newacronym{vlc}{VLC}{Variable Length Codes}
\newacronym{cabac}{CABAC}{Context Adaptative Binary Arithmetic Coding}
\newacronym{jvt}{JVT}{Joint Video Team}
\newacronym{psnr}{PSNR}{Peak Signal to Noise Ratio}
\newacronym{mm}{MM}{Matrix Multiplication}
\newacronym[sort=mse]{ems}{$MSE$}{Mean Square Error}
\newacronym{qp}{QP}{Quantization Parameter}
\newacronym{cpu}{CPU}{Central Processing Unit}
\newacronym[plural=Field-Programmable Gate Arrays (FPGAs)]{fpga}{FPGA}{Field-Programmable Gate Array}
\newacronym[plural=Application Specific Integrated Circuits (ASICs)]{asic}{ASIC}{Application Specific Integrated Circuit}
\newacronym{vhdl}{VHDL}{Very High Speed Integrated Circuit (VHSIC) Hardware Description Language (HDL)}
%\newacronym{jpeg}{JPEG}{Joint Photographic Experts Group}


\newglossaryentry{rgb}{%
  name={RGB},
  description={Color space based on the addition of Red, Green and Blue components for complex color representation}
}

\newglossaryentry{codec}{
  name={Codec},
  description={Encoder-Decoder. Also referred to the method of compressing and decompressing a video sequence}
}

\newglossaryentry{interlacing}{
  name={Interlaced scanning},
  description={Technique used by televisions for broadcasting and displaying, where only odd or even numbered lines of a frame are transmitted/displayed at a time, alternately}
}

\newglossaryentry{progressive}{
  name={Progressive scanning},
  description={Technique used by more recent screens, where each frame is displayed as a whole, from top to bottom, and left to right}
}

\newglossaryentry{VP89}{
  name={VP8/VP9},
  description={Open-format video codecs developed by Google, released in 2008 and 2013, respectively}
}

\newglossaryentry{libaom}{
  name={libaom},
  description={Reference software for AV1, released by Google in June 2018}
}

\newglossaryentry{JPEG}{
  name={JPEG},
  description={Still image compression format, developed by the Joint Photographic Experts Group (JPEG)}
}

\newglossaryentry{pixel}{
  name={Pixel},
  description={Picture Element}
}

\newglossaryentry{h264}{
  name={H.264/AVC},
  description={Previous state of the art video codec from \gls{jvt}, released in 2007. As to the writing of this work, it is the most used video compression algorithm.}
}

\newglossaryentry{bdrate}{
  name={Bjontegaard-Delta rate (BD-rate)},
  description={Objective quality metric, that evaluates the bitrate savings according to the obtained PSNR}
}

\newglossaryentry{tld:round}{
  name={$\mathbf{\nint*{x}}$},
  description={Round to the nearest integer},
  sort=round,
  type=nomenclature
}

\newglossaryentry{tld:floor}{
  name={$\mathbf{\nfloor*{x}}$},
  description={Round to the nearest integer, lower than $x$},
  sort=floor,
  type=nomenclature
}

\newglossaryentry{tld:ceil}{
  name={$\mathbf{\nceil*{x}}$},
  description={Round to the nearest integer, greater than $x$},
  sort=ceil,
  type=nomenclature
}

\newglossaryentry{tld:shiftl}{
  name={$x<<K$},
  description={Shift $x$, $k$ bits to the left},
  sort=shiftl,
  type=nomenclature
}

\newglossaryentry{tld:shiftr}{
  name={$x>>K$},
  description={Shift $x$, $k$ bits to the right},
  sort=shiftr,
  type=nomenclature
}

\newglossaryentry{tld:vector}{
  name={$\vec{x}$},
  description={$N$th dimension vector},
  sort=vectora,
  type=nomenclature
}

\newglossaryentry{tld:matrix}{
  name={$\mathbf{X}$},
  description={Matrix},
  sort=vectoraa,
  type=nomenclature
}

\newglossaryentry{tld:transformed}{
  name={$\mathcal{\vec{X}}$},
  description={Input vector $\vec{x}$ in the \emph{Transform} domain},
  sort=vectorb,
  type=nomenclature
}

\newglossaryentry{tld:restvector}{
  name={$\underset{*}{\vec{x}}$},
  description={Restored version of vector $\vec{x}$},
  sort=vectorc,
  type=nomenclature
}

\newglossaryentry{tld:min}{
  name={$\min{\mathbf{X}}$},
  description={Smallest value of matrix $\mathbf{X}$},
  sort=min,
  type=nomenclature
}

\newglossaryentry{tld:max}{
  name={$\max{\mathbf{X}}$},
  description={Largest value of matrix $\mathbf{X}$},
  sort=max,
  type=nomenclature
}

\glsaddall
