%%%%%%%%%%%%%%%%%%%%%%
%%%%%% PACKAGES %%%%%%
%%%%%%%%%%%%%%%%%%%%%%

%%%%%%%%%%%%%%%%%
%%% Mandatory %%%
%%%%%%%%%%%%%%%%%
\documentclass[11pt,twoside,a4paper]{report}
\usepackage[DETI,newLogo]{uaThesis}

%%%%%%%%%%%%%%%%
%%% Optional %%%
%%%%%%%%%%%%%%%%
\usepackage[english]{babel}
\usepackage{hyperref}
\usepackage{amsmath}
\usepackage{amssymb}
\usepackage{xspace}% used by \sigla
\usepackage{lipsum}
\usepackage[backend=biber, refsection=chapter, sorting=none, url=false]{biblatex}
\addbibresource{newBib.bib}
\usepackage{csquotes}
\usepackage[acronym,nonumberlist]{glossaries}
\newglossary[tlg]{nomenclature}{tld}{tdn}{Nomenclature}
\usepackage{xcolor}
\usepackage{todonotes}
\usepackage[bottom]{footmisc}
\usepackage{amsmath}
\usepackage{xfrac}
\usepackage{tikz}
\usepackage{mathdots}
\usepackage{yhmath}
\usepackage{cancel}
\usepackage{color}
\usepackage{siunitx}
\usepackage{array}
\usepackage{multirow}
\usepackage{amssymb}
\usepackage{gensymb}
\usepackage{tabularx}
\usepackage{booktabs}
\usetikzlibrary{fadings,shapes,arrows,chains,backgrounds,calc,arrows,arrows.meta,shadows,shapes.geometric, fit, positioning, circuits.logic.US, circuits}
\usepackage{caption,subcaption}
\usepackage{graphicx, epstopdf}
\usepackage{mathtools, nccmath}
\usepackage{pdflscape}
\usepackage{afterpage}
\usepackage{url}
\PassOptionsToPackage{hyphens}{url}
\usepackage[toc,page,title]{appendix}
\usepackage{listings}
\lstset{
basicstyle=\small\ttfamily,
columns=fullflexible,
breaklines=true
}
\usepackage{pgfplots}
% and optionally (as of Pgfplots 1.3):
\pgfplotsset{compat=newest}
\pgfplotsset{plot coordinates/math parser=false}
\makeatletter
\g@addto@macro\bfseries{\boldmath}
\makeatother
\usepackage{array}
\newcolumntype{L}{>{\centering\arraybackslash}m{0.35\textwidth}}
\usepackage{units}

%%%%%%%%%%%%%%%%%%%%%%%
% Tikz library and set
\include{tikzlib}

%%%%%%%%%%%%%%%%%%%%%%%
%%%% Header makeup %%%%
%%%%%%%%%%%%%%%%%%%%%%%

%\usepackage[Sonny]{fncychap}
%\ChNameVar{\Huge\bfseries}
%\ChNumVar{\Huge\bfseries}
%\ChTitleVar{\LARGE\bfseries}
\usepackage{color}
\definecolor{gray75}{gray}{0.75}
\usepackage{titlesec}

\titlespacing*{\chapter}{0pt}{-10pt}{60pt}

\titleformat{\chapter}[display] % shape
{\bfseries\LARGE} % format
{\MakeUppercase{\chaptertitlename} \Huge\thechapter} % label
{0.5ex} % sep
{\titlerule\vspace{1ex}\filleft}
[\vspace{1ex}\titlerule]

\newcommand{\dgrayvline}{\raisebox{-3.5mm}{\tikz\draw[darkgray, very thick] (0,0) -- ++(12mm,0) -- ++(0,10mm) -- ++(-12mm,0);}}

\titleformat{\section}[hang]
{\Large\bfseries}
{\thesection\hspace{-8.75mm}\dgrayvline\hspace{10pt}}
{0pt}
{\Large\bfseries}

\newcommand{\grayvline}{\raisebox{-3.5mm}{\tikz\draw[gray, very thick] (0,0) -- ++(6mm,0) -- ++(0,10mm) -- ++(-6mm,0);}}

\titleformat{\subsection}[hang]
{\large\bfseries}
{\thesubsection\hspace{-2.75mm}\grayvline\hspace{10pt}}
{0pt}
{\large\bfseries}

\newcommand{\lgrayvline}{\raisebox{-3.5mm}{\tikz\draw[lightgray,very thick] (0,0) -- ++(3mm,0) -- ++(0,10mm) -- ++(-3mm,0);}}

\titleformat{\subsubsection}[hang]
{\normalsize\bfseries}
{\thesubsubsection\hspace{0pt}\lgrayvline\hspace{10pt}}
{0pt}
{\normalsize\bfseries}

\newcommand{\textsep}{\noindent\makebox[\linewidth]{\resizebox{0.3333\linewidth}{1pt}{$\bullet$}}\bigskip}


%%%%%%%%%%%%%%%%%%%%%%
%%%%%%% MACROS %%%%%%%
%%%%%%%%%%%%%%%%%%%%%%

%%%%%%%%%%%%
% TOS
\setcounter{tocdepth}{2}
\setcounter{secnumdepth}{3}
% optional (comment to used default)
%   horizontal line to separate floats (figures and tables) from text
%\def\topfigrule{\kern 7.8pt \hrule width\textwidth\kern -8.2pt\relax}
%\def\dblfigrule{\kern 7.8pt \hrule width\textwidth\kern -8.2pt\relax}
%\def\botfigrule{\kern -7.8pt \hrule width\textwidth\kern 8.2pt\relax}

% custom macros (could also be defined using \newcommand)
\def\I{\mathtt{i}}         % one possible way to represent $\sqrt{-1}$
\def\Exp#1{e^{2\pi\I #1}}  % argument inside braces, i.e., "{}"
\def\EXP#1.{e^{2\pi\I #1}} % argument finishes when a full stop is encountered, i.e., "."
\def\sigla{\LaTeX\xspace}  % use as "blabla \sigla blabla (no need to do "blabla \sigla\ blabla"

\def\AddVMargin#1{\setbox0=\hbox{#1}%
                  \dimen0=\ht0\advance\dimen0 by 2pt\ht0=\dimen0%
                  \dimen0=\dp0\advance\dimen0 by 2pt\dp0=\dimen0%
                  \box0}   % add extra vertical space above and below the argument (#1)
\def\Header#1#2{\setbox1=\hbox{#1}\setbox2=\hbox{#2}%
           \ifdim\wd1>\wd2\dimen0=\wd1\else\dimen0=\wd2\fi%
           \AddVMargin{\parbox{\dimen0}{\centering #1\\#2}}} % put #1 on top #2

%%%%%%%%%%%%
% Mine
\def\ThesisYear{\the\year}
\def\myName{Miguel Oliveira Inocêncio}
\def\TituloTese{Co-processador da Transformada para AV1}
\def\ThesisTitle{AV1 Transform Co-Processor}

\setlength\bibitemsep{2.5\itemsep}        % Aumenta espaçamento entre entradas da bibliografia

\emergencystretch=20em                     % Corrigir overfull hbox

\setlength{\parskip}{0ex}

\newcommand{\figwidth}{0.6\textwidth}

\renewcommand{\arraystretch}{1.2}
\setlength{\tabcolsep}{12pt}

\DeclarePairedDelimiter{\nint}\lfloor\rceil
\DeclarePairedDelimiter{\nfloor}\lfloor\rfloor
\DeclarePairedDelimiter{\nceil}\lceil\rceil

\interfootnotelinepenalty=10000         % Change priority for footnotes

\newcommand{\todor}[1]{\todo[inline, color=red!40]{#1}}
\newcommand{\pathcos}[6]{\path (#1.east) to node [pathcos, color=#5, #6] {$\sfrac{#3\pi}{#4}$} (#2.west)}

\newcommand{\restor}[1]{\underset{*}{#1}}

%%%%%%%%%%%%%%%%%%%%
% Carregar Glossário
\makenoidxglossaries
\loadglsentries{glossary}

%%%%%%%%%%%%%%%%%%%%%%%%%%%%%%%%%%%
%%%%%%%%% BEGIN DOCUMENT %%%%%%%%%%
%%%%%%%%%%%%%%%%%%%%%%%%%%%%%%%%%%%
\begin{document}

%%%%%%%%%%%%%%%%%%%%%%%%%%%%%%%%%%%%%%%%%%%%%%%%%%%%%%%%%%%%%%%%
% Capa, contracapa, Júri, agradecimentos, palavras chave, resumo
% CoverPage
\TitlePage

  \HEADER{\BAR}
         {\ThesisYear}
  \TITLE{\myName}
        {\TituloTese}
        {\ThesisTitle}
\EndTitlePage
\titlepage\ \endtitlepage % empty page


%
% Initial thesis pages
%

\TitlePage
  \HEADER{}{\ThesisYear}
  \TITLE{\myName}
        {\TituloTese}
        {\ThesisTitle}
  \vskip 15mm
  \TEXT{}
       {Dissertação de Mestrado apresentada à Universidade de Aveiro, para obtenção do grau de Mestre em Engenharia Eletrónica e de Telecomunicações, sob orientação científica do Professor Doutor António Navarro, Professor auxiliar do Departamento de Eletrónica, Telecomunicações e Informática da  Universidade de Aveiro, com colaboração do Professor Luciano Agostini e do Video Technology Research Group - ViTech - da Universidade Federal de Pelotas - UFPel.}
\EndTitlePage
\titlepage\ \endtitlepage % empty page

\TitlePage
  \vspace*{55mm}
  \TEXT{\textbf{o júri~/~the jury\newline}}
       {}
  \TEXT{presidente~/~president}
       {\textbf{Prof. Dr. Armando José Formoso de Pinho
       }\newline {\small
        Professor Associado com Agregação da Universidade de Aveiro}}
  \vspace*{5mm}
  \TEXT{vogais~/~examiners committee}
       {\textbf{Prof. Dr. Pedro António Amado Assunção}\newline {\small
        Professor Coordenador, Escola Superior de Tecnologia e Gestão de Leiria do Instituto Politécnico de Leiria (Arguente Principal)}}
  \vspace*{5mm}
  \TEXT{}
       {\textbf{Prof. Dr. António José Nunes Navarro Rodrigues}\newline {\small
       Professor Auxiliar da Universidade de Aveiro (Orientador)}}
\EndTitlePage
\titlepage\ \endtitlepage % empty page

\TitlePage
  \vspace*{55mm}
  \TEXT{\textbf{agradecimentos~/\newline acknowledgements}}
       {À minha família por todo o apoio que deram ao longo destes 5 longos anos.}
  \TEXT{}
       {A todos os amigos e companheiros por terem feito parte destes 5 curtos anos.}
  \TEXT{}
       {Ao João, Edgar, Correia, Kevin, Diogo e à República de que vamos sempre fazer parte.}       
  \TEXT{}
       {Ao Luís, Simão, Soares, Santos e ao segundo piso da Biblioteca, que conhecemos melhor que ninguém.}
\EndTitlePage
\titlepage\ \endtitlepage % empty page

\TitlePage
  \vspace*{55mm}
  \TEXT{\textbf{Palavras-Chave}}
       {Compressão de Vídeo, AV1, Transformadas, DCT, FPGA}
  \TEXT{\textbf{Resumo}}
       {Esta dissertação apresenta o estudo efetuado sob o formato de compressão de vídeo \emph{AV1}, no qual se baseia o desenvolvimento de arquiteturas para o módulo da Transformada, nomeadamente, a \emph{DCT}. Numa primeira fase foram abordados métodos de otimização do software de referência, tendo em conta as características identificadas nos testes de codificação. Com as modificações efetuadas, foram obtidos ganhos de $3\%$ no tempo de codificação em relação ao encoder original.}
  \TEXT{}
       {O algoritmo implementado em software foi posteriormente descrito em VHDL, tendo-se obtido duas variantes implementáveis, sendo que uma destas foi sintetizada e testada numa FPGA Nexys 4, tendo se obtido uma utilização de 78.93\% e um consumo de 50 mW. No kit de hardware onde foi implementada, esta arquitetura tem capacidade para processar vídeo FHD até 71 frames por segundo.}
\EndTitlePage
\titlepage\ \endtitlepage % empty page

\TitlePage
  \vspace*{55mm}
  \TEXT{\textbf{Keywords}}
       {Video Coding, AV1, Transform Coding, DCT, FPGA}
  \TEXT{\textbf{Abstract}}
       {This dissertation presents a study made about the \emph{AV1} video coding standard, on which the development of an architecture for the \emph{Transform} stage, namely the \emph{DCT}, was based. Firstly, some optimization methods were implemented to the reference encoder, accordingly to some exploitable characteristics found on the encoding tests. With the developed software modifications, a gain of $3\%$ on encoding time was achieved, comparing to the original encoder.}
  \TEXT{}
       {The algorithm implemented in software was then described in VHDL, obtaining two implementable architectures, one of which was synthesized and tested on a Nexys 4 FPGA, obtaining 79.93\% utilization and a 50 mW consumption. On the hardware kit on which it was implemented, this architecture is able to process FHD video up to 71 frames per second.}
\EndTitlePage
\titlepage\ \endtitlepage % empty page



%%%%%%%%%%%%%%%%%%%%%%%%%%%%%%%%%%%%%
% Tables of contents, of figures, ...
\pagenumbering{roman}
\tableofcontents

\cleardoublepage
\listoffigures
\addcontentsline{toc}{chapter}{\listfigurename}

\cleardoublepage
\listoftables
\addcontentsline{toc}{chapter}{\listtablename}

\cleardoublepage
\printnoidxglossary[type=\acronymtype]
\addcontentsline{toc}{chapter}{\acronymname}

\cleardoublepage
\printnoidxglossary
\addcontentsline{toc}{chapter}{\glossaryname}

\cleardoublepage
\printnoidxglossary[type=nomenclature,title=Nomenclature]
\addcontentsline{toc}{chapter}{Nomenclature}

%%%%%%%%%%%%%%%%%%%%%%%%%%%%%%%%%%%%%
% Introduction
\cleardoublepage
\pagenumbering{arabic}
\chapter{Introduction}

\section{\textcolor{red}{Context}}
\todo[inline,color=orange]{* Desenvolvimento da qualidade de vídeo ao consumidor final}
\todo[inline,color=red]{* Depêndencia das baterias para aplicações móveis}
\todo[inline,color=red]{* Hardware dedicado melhora performance e consumo energético}

Since the spark of television research in 1887 \cite{schubinWhatSparkedVideo2017}, a tremendous investment has been put into providing the user with better quality video

\printbibliography[heading=subbibliography]
\addcontentsline{toc}{section}{References}


%%%%%%%%%%%%%%%%%%%%%%%%%%%%%%%%%%%%%
% Basics of Video Encoding
%\cleardoublepage
\chapter{Basics of Video Compression}

%In oder to have a better understanding of the video 

%%%%%%%%%%%%%%%%%%%%%%%%%%%%%%%%%%%%%%%%%%%%%%%%%%%%%%%%%%%%%%%%%%%%%%%%%%%%%%
\section{Human Visual System}

\todo[inline,color=red!40]{*Essency of video compression relies on making changes the image without serious perception by the user}
\todo[inline,color=red!40]{*Eye Functioning}
\todo[inline,color=red!40]{*"Known Issues" (lower perception to chroma, high frequencies, etc)}
\todo[inline,color=red!40]{*Opportunity to explore various types of redundancies to the image}

%%%%%%%%%%%%%%%%%%%%%%%%%%%%%%%%%%%%%%%%%%%%%%%%%%%%%%%%%%%%%%%%%%%%%%%%%%%%%%
\section{Redundancy Exploitation}

\todo[inline,color=red!40]{*Types of redundancies (Temporal, Statistical and Coding)}
\todo[inline,color=red!40]{*Color subsampling}
\todo[inline,color=red!40]{*Intra-prediction}
\todo[inline,color=red!40]{*Inter-prediction}
\todo[inline,color=red!40]{*Transform and Quantization}
\todo[inline,color=red!40]{*Entropy Coding}

%%%%%%%%%%%%%%%%%%%%%%%%%%%%%%%%%%%%%%%%%%%%%%%%%%%%%%%%%%%%%%%%%%%%%%%%%%%%%%
\section{Basic Video Compression/Decompression System}

\todo[inline,color=red!40]{*Encoder Model}
\todo[inline,color=red!40]{*Decoder Model}


\printbibliography[heading=subbibliography]
\addcontentsline{toc}{section}{References}


%%%%%%%%%%%%%%%%%%%%%%%%%%%%%%%%%%%%%
% AV1 Video Codec
\cleardoublepage
\chapter{Video Compression Systems}\label{chap:av1}

%%%%%%%%%%%%%%%%%%%%%%%%%%%%%%%%%%%%%%%%%%%%%%%%%%%%%%%%%%%%%%%%%%%%%%%%%%%%%%
\section{Basic Principles}

\emph{Video Compression Systems} have been in development for approximately forty years, with the first video codec, \emph{H.120}, being released in 1984. It was composed of basic operations, which didn't correlate to good compression performances. This has lead to a quick downfall of its usage, being aggravated by the release of the \emph{H.261} standard by 1984.

However, the building blocks on which later standards were based are the same as in the first generations, i.e., the strategies implemented on newer standards exploit the same \emph{redundancies} as previous, less efficient, codecs. 

This way, to have a better understanding of the functioning behind video codecs, the mentioned redundancies, and respective origins, are presented. Most of such have origin on the way humans perceive vision, being this the first topic of this chapter.

%%%%%%%%%%%%%%%%%%%%%%%%%%%%%%%%%%%%%%%
\subsection{Human Visual System}

%\todo[inline,color=green!40]{*Essency of video compression relies on making changes the image without serious perception by the user}
%\todo[inline,color=green!40]{*Eye Functioning}
%\todo[inline,color=red!40]{*"Known Issues" (lower perception to chroma, high frequencies, etc)}
%\todo[inline,color=red!40]{*Opportunity to explore various types of redundancies to the image}

\nocite{gonzalezDigitalImageProcessing2018}

Most of the compressed/decompressed video nowadays is directed to content visualization by consumers, with the exception of some network-driven image processing applications, \textcolor{red}{e.g. \dots}. 
\todo[inline,color=red!40]{example of across network video processing}
Therefore, the compression of video sequences has the intent of making changes to the original data, without serious impact to the users' perception. This process is mentioned as the removal of the \emph{Psychovisual redundancy} \cite{shiImageVideoCompression2008}. Therefore, a basic understanding of the visual system can clarify many of the design choices made in video compression applications, and why their use doesn't present much impact on the quality of the image, while greatly reducing its size.

The image perception starts in the human eye, represented in figure \ref{fig:eye}. Its different constituents accomplish different tasks, from focusing, to aperture control. Although their importance to the overall functioning of the eye, the part that matters most to the focus of this work is the innermost membrane, the retina.

\begin{figure}[h]
    \centering
    \missingfigure{Eye Scheme}
    \caption{Representation of an human eye}
    \label{fig:eye}
\end{figure}

Once the desired image is properly focused by the lens, an inverse version of it is shined on the aforementioned membrane, which is covered by two types of light sensitive cells, the \emph{cones} and \emph{rods}, which transform the observable image into a series of pulses, that get subsequently processed.

The cones are highly sensitive to color, being responsible for the \emph{photonic} or \emph{bright-light} vision. There are three different types of cones, corresponding to the wavelength they are susceptible to. These are the \emph{S}, \emph{M} and \emph{L} cones, being sensitive to, approximately, the blue, green and red light, respectively, making a somewhat similar capture to the RGB color system.

On the other side, rods aren't stimulated by bright light, being more active on low illumination levels. This aspect makes them responsible for giving a rough overview of the field of view. This is called as \emph{scotopic} or \emph{dim-light} vision. These cells are spread more broadly across the retina, while to the cones, which is also observable in the number of cells (approximately 6 million cones, to 100 million rods).

From this, it's already observable that the human visual system is more sensitive to differences on the luminosity, than to the color of an object \cite{mullenContrastSensitivityHuman1985}, which is a starting point for compressing video, as will be shown later in this chapter. However, many other opportunities come from the processing of the nerve signals, and the \emph{psychovisual} perception that follows.

Although more sensitive to \emph{luminance}, there is a threshold to which the difference between two objects --- $\Delta I$ --- can't be discerned. This relation is mentioned as \emph{contrast sensitivity function}, which is roughly approximated with the \emph{Weber's Law}

\begin{equation}
    \frac{\Delta I}{I}\approx constant
\end{equation}

Analyzing this equation, it's possible to conclude that the darker an object is, the lower the difference in luminance needs to be to distinguish another object. Also, darker images tend to be more susceptible to compression artifacts.

Besides the luminance values, the spatial and temporal frequencies also represent an important role in the perception of such errors. 

The image \ref{fig:noise} gives an example of the dependency with spatial frequency. The first image \ref{subfig:noiseOri} represents the original image, which got added with white gaussian noise (\emph{WGN}) on figure \ref{subfig:noise}. As it is observable, these artifacts are less noticeable on the highly detailed areas than in the smooth ones.

\begin{figure}[h]
    \centering
    \begin{subfigure}[c]{\textwidth}
        \centering
        \missingfigure{Original Image}
        \caption{Original Image}
        \label{subfig:noiseOri}
    \end{subfigure}
    \begin{subfigure}[c]{\textwidth}
        \centering
        \missingfigure{Image with AWGN}
        \caption{Image with added WGN}
        \label{subfig:noise}
    \end{subfigure}
    \caption{Example of the effect of added noise on figure}
    \label{fig:noise}
\end{figure}

Temporal frequency dependency, although more challenging to exemplify, is easily understandable. On a sequence of frames with fast camera or subject movements, the human eye doesn't have the ability to track details or other artifacts, while in slow moving scenes, it can easily identify errors.

These are some of the "\emph{flaws}" of the human visual system, that get exploited during the compression of video. However, other \emph{redundancies}, inherent from the captured images themselves contribute to the reduction of the video size, as will be described in the following sections.

%%%%%%%%%%%%%%%%%%%%%%%%%%%%%%%%%%%%%%%
\subsection{Redundancy Exploitation}

\todo[inline,color=red!40]{*Types of redundancies (Temporal, Statistical and Coding)}
\todo[inline,color=red!40]{*Color subsampling}
\todo[inline,color=red!40]{*Intra-prediction}
\todo[inline,color=red!40]{*Inter-prediction}
\todo[inline,color=red!40]{*Transform and Quantization}
\todo[inline,color=red!40]{*Entropy Coding}

%%%%%%%%%%%%%%%%%%%%%%%%%%%%%%%%%%%%%%%
\subsection{Basic Video Compression/Decompression System}

\todo[inline,color=red!40]{*Encoder Model}
\todo[inline,color=red!40]{*Decoder Model}

%%%%%%%%%%%%%%%%%%%%%%%%%%%%%%%%%%%%%%%%%%%%%%%%%%%%%%%%%%%%%%%%%%%%%%%%%%%%%%
\section{Previous Standards}

\todo[inline,color=red!40]{*Previous generations}

%%%%%%%%%%%%%%%%%%%%%%%%%%%%%%%%%%%%%%%%%%%%%%%%%%%%%%%%%%%%%%%%%%%%%%%%%%%%%%
\section{AV1}

\todo[inline,color=red!70]{\textbf{Review \textit{AV1 Bitstream and Decoding Process}}}
\todo[inline,color=red!40]{*Development Process}
\todo[inline,color=red!40]{*AOMedia companies}
\todo[inline,color=red!40]{*Comparation with past generations}
\todo[inline,color=red!40]{*Introduction of modules not present on other video codecs}
\todo[inline,color=red!40]{*Block diagram}

\input{Sections/3AV1/Diagrams/av1block.tex}

%%%%%%%%%%%%%%%%%%%%%%%%%%%%%%%%%%%%%%%%%%%%%%%%%%%%%%%%%%%%%%%%%%%%%%%%%%%%%%
\section{Performance Analysis}

\todo[inline,color=red!40]{*Compression gains}
\todo[inline,color=red!40]{*Quality assessment}
\todo[inline,color=red!40]{*Complexity (general/modules) and timing issues}

%%%%%%%%%%%%%%%%%%%%%%%%%%%%%%%%%%%%%
% AV1 Video Codec
\cleardoublepage
\chapter{Video Coding Transforms}

%%%%%%%%%%%%%%%%%%%%%%%%%%%%%%%%%%%%%%%%%%%%%%%%%%%%%%%%%%%%%%%%%%%%%%%%%%%%%%
\section{\textcolor{red}{Introduction}}

\textcolor{red}{As mentioned previously,} the basic principle behind the compression of video, is the reduction of inter-pixel/inter-symbol correlation. Therefore, all of the integral blocks of a video compression system output a better compressed symbol than its input. One of such blocks is the \emph{Transform} 
\todo{as seen in ...}
,which is the focus of this work.

\todo[inline,color=red!40]{Verify accordance with previous chapter}

The technique implemented by this process relies on the energy compaction in the frequency domain, to reduce the correlation within a frame block, i.e. the input of the Transform block is evaluated on its main frequencies --- the \emph{transform coefficients}, on a spatial and/or temporal domain, similarly to the process executed on an \gls{fft}. Once each block is quantized on these coefficients, the compression is made with the removal of the least significant, on the \emph{Quantization}.

Besides its use in video coding, many other areas use some sort of component transformation, namely in audio compression, voice identification, et al.

A useful interpretation of this process is to see it as the decomposition of the input as a set of basis vectors (1D transforms) or images/matrices (2D transforms). The transformation outputs , $y_i$, can be seen as the weights of each basis vector/image, $\vec{e_i}$, that summed return the input, $\vec{z}$, i.e.

\begin{equation}
    \vec{z} = \sum_{i=1}^{N} y_i \vec{e_i}
\end{equation}
which means that the coefficients are related to the amount of correlation between the input and each basis component, and can be obtained with the inner product of the input and each basis vector \cite[sec. 4.1.4 \& 4.2.2]{shiImageVideoCompression2008}.

\begin{equation}
    y_i = \vec{e_i}^T \vec{z}
\end{equation}




%%%%%%%%%%%%%%%%%%%%%%%%%%%%%%%%%%%%%%%%%%%%%%%%%%%%%%%%%%%%%%%%%%%%%%%%%%%%%%
\section{\textcolor{red}{Background}}

%%%%%%%%%%%%%%%%%%%%%%%%%%%%%%%%%%%%%%%%%%%%%%%%%%%%%%%%%%%%%%%%%%%%%%%%%%%%%%
\section{\textcolor{red}{Used Transformation Kernels}}

\printbibliography[heading=subbibliography]
\addcontentsline{toc}{section}{References}

%%%%%%%%%%%%%%%%%%%%%%%%%%%%%%%%%%%%%
% Developed Architecture
\cleardoublepage
\chapter{Developed Architectures}

%%%%%%%%%%%%%%%%%%%%%%%%%%%%%%%%%%%%%%%%%%%%%%%%%%%%%%%%%%%%%%%%%%%%%%%%%%%%%%
\section{Objectives and Workflow}

The previous chapter presented some characteristics of the current state of \emph{libaom}'s \emph{Transform} stage which might compromise its performance, the most relevant being the unnecessary flexibility in the representation of cosine approximations.

In order to undertake these opportunities, and improve the overall encoder performance, new architectures for the studied stage were developed.

The developed implementations tackled the forward \emph{DCT}, since it was the \emph{kernel} that would have the most impact on encoder performance. As the \emph{IDCT} is shared between encoder and decoder, and due to the added complexity, no changes were done to this block, as it acts with accordance with the established standard, as mentioned previously. 

%%%%%%%%%%%%%%%%%%%%%%%%%%%%%%%%%%%%%%%%%%%%%%%%%%%%%%%%%%%%%%%%%%%%%%%%%%%%%%
\section{Matrix Multiplication Implementation}

The first test was the application of the simplest integer \emph{DCT}, done by the multiplication of the input vector by a scaled up version of the transform matrix, $\mathbf{F}$, firstly shown in Equation \ref{eq:DCT2}. 

The original integer transform matrix is shown in equation \ref{eq:matscale}.

\begin{equation} \label{eq:matscale}
    \begin{gathered}
        \mathbf{F}_{x,u} = \beta(u)\cos\left(\frac{(2x+1)u\pi }{2L}\right)\,0\leq u,x < L \\
        \Downarrow \\
        \mathbf{F} = \sqrt{\frac{2}{L}}  \begin{bmatrix}
            \sqrt{\frac{1}{2}}                                  & \sqrt{\frac{1}{2}}                                & \dots & \sqrt{\frac{1}{2}} \\
            \cos\left(\frac{\pi}{2L}\right)    & \cos\left(\frac{3\pi}{2L}\right) & \dots & \cos\left(\frac{(2(L-1)+1)\pi}{2L}\right) \\
            \vdots     & \vdots     & \ddots & \vdots       \\
            \cos\left(\frac{(L-1)\pi}{2L}\right)    & \cos\left(\frac{3(L-1)\pi}{2L}\right) & \dots & \cos\left(\frac{(2(L-1)+1)(L-1)\pi}{2L}\right) \\
        \end{bmatrix} 
    \end{gathered}
\end{equation}

As mentioned previously, the floating point coefficients bring a number of disadvantages on an hardware implementation, from increased calculation overheads, to encoder/decoder mismatches. 

In order to address these problems, a scale and rounding operation was performed, as shown in Equation \ref{eq:matscale}, where $K$ represents the number of bits of the scaled coefficients.

\begin{equation} \label{eq:matscale}
    \nint*{\mathbf{F}_K}   = \nint*{2^K \mathbf{F}}
\end{equation}

However, due to the rectangular block sizes allowed in \emph{AV1}, the factor $\sqrt{\frac{2}{L}}$ isn't considered in the kernels themselves. Instead, the transformed outputs get scaled at a later stage. This way, the implemented transform matrix is

\begin{equation} \label{eq:matscale2}
    \begin{gathered}
        \nint*{\mathbf{F}_K}   = \nint*{2^K \sqrt{\frac{L}{2}}\mathbf{F}} \\
        = \nint*{2^K \begin{bmatrix}
                        \sqrt{\frac{1}{2}}                                  & \sqrt{\frac{1}{2}}                                & \dots & \sqrt{\frac{1}{2}} \\
                        \cos\left(\frac{\pi}{2L}\right)    & \cos\left(\frac{3\pi}{2L}\right) & \dots & \cos\left(\frac{(2(L-1)+1)\pi}{2L}\right) \\
                        \vdots     & \vdots     & \ddots & \vdots       \\
                        \cos\left(\frac{(L-1)\pi}{2L}\right)    & \cos\left(\frac{3(L-1)\pi}{2L}\right) & \dots & \cos\left(\frac{(2(L-1)+1)(L-1)\pi}{2L}\right) \\
                    \end{bmatrix} 
                }
    \end{gathered}
\end{equation}

This way, the  transformed outputs are calculated through

\begin{equation}
    \vec{\mathcal{G}} = \left[\nint*{\mathbf{F}_K} \vec{g}\right]>>K
\end{equation}

For an $L$ length vector, the calculation of the transformed vector implies $L^2$ additions and $L^2$ multiplications, which leads to the main disadvantage of such implementation. For larger vectors, this operation becomes too demanding in terms of memory and complexity.

One other negative aspect of such implementation is that, due to the variation of the transform matrix's coefficients, the obtained error in the rounding and scaling operation also varies with the vector size. The quantization\footnote{Here, quantization refers to the scaling and rounding operation, and not to the the \emph{Q} stage in an encoder.} error, $\Delta_K$, can be calculated as

\begin{equation}
    \Delta_K = \frac{\max{\left(\sqrt{\frac{L}{2}}\mathbf{F}\right)} - \min{\left(\sqrt{\frac{L}{2}}\mathbf{F}\right)}}{2^K}
\end{equation}

As it was proven in the previous Chapter that the number of bits in the cosine representation wouldn't greatly impact the quality of the video, the developed architectures used 8 bits for the scaling operation, as to decrease the overhead of the implemented multiplications and shifts. The impact of this choice was evaluated at a later stage.

To evaluate the performance of this first implementation, there was performed a test that measured and compared the elapsed time for both the described architecture, and the corresponding equivalent from \emph{aomenc}. This test injected a fixed sequence of input vectors into each of the \emph{DCT}'s, scaled the vectors in the same manner as the encoder, and then re-transformed the obtained vectors with \emph{AV1}'s \emph{IDCT}. This sequence of operations was timed, and the corresponding results are presented at Table \ref{tab:dcttime}.

\begin{table}[!htpb]
    \centering
    \begin{tabular}{ccc} \toprule
        \multirow{2}{*}{\textbf{Vector Size}} &     \multicolumn{2}{c}{\textbf{Execution Time (ms)}} \\
         &      \textbf{aomenc's} &      \textbf{MM} \\ \toprule
        4 &    145 &      98 $(-32.5\%)$ \\ \hline
        8 &    319 &      215 $(-32.6\%)$ \\ \hline
        16 &   787 &      583 $(-25.9\%)$ \\ \hline
        32 &   1917 &     1606 $(-16.2\%)$  \\ \hline
        64 &   4999  &    5696 $(+14.0\%)$  \\ 
        \bottomrule
    \end{tabular}
    \caption{Comparison of execution time between \emph{aomenc}'s DCT and the described implementation.}
    \label{tab:dcttime}
\end{table}

From these results it's easily observable why the currently implemented transforms follow the \emph{butterfly} scheme. Although from sizes $4$ to $32$, the proposed implementation is faster than the current version of \emph{libaom}, the largest transform is slower. This factor, added to the error variation from the scaling operation makes this implementation quite damaging for the overall encoder performance, especially on a constant quality objective, as shown in Table \ref{tab:multresults}. Here, there are presented the timing results of an encoding test, where one encode was made with the standard \emph{aomenc}, the other had the proposed matrix multiplication \emph{DCT}'s. The test encoded the first 15 frames of the \emph{Parkrun} HD sequence, with two different quality objectives. After compression, the encoded video was decoded with \emph{aomdec}, calculating the PSNR of the output video.

\begin{table}[!htpb]
    \centering
    \begin{tabular}{cccc} \toprule
        \multicolumn{2}{c}{\multirow{2}{*}{\textbf{cq-level}}} &     \multicolumn{2}{c}{\textbf{Execution Time (ms)}} \\
        &   &   \textbf{aomenc's} &      \textbf{MM} \\ \toprule
         \multirow{3}{*}{\textbf{60}}   & \textbf{Total time (s)}       & 466.5     & 530.8 \\
                                        & \textbf{\emph{Trans.} time (s)}    & 45.0      & 104.2 \\
                                        & \textbf{PSNR (dB)}            & 32.39     & 32.38 \\ \hline
         \multirow{3}{*}{\textbf{5}}    & \textbf{Total time (s)}       & 814.1     & 835.3 \\
                                        & \textbf{\emph{Trans.} time (s)}    & 60.4      & 98.4 \\
                                        & \textbf{PSNR (dB)}            & 34.88     & 34.86 \\                                        
         \bottomrule
    \end{tabular}
    \caption{\emph{aomenc}'s encoding time with original vs implemented \emph{DCT}.}
    \label{tab:multresults}
\end{table}

As it is observable, to maintain a similar encoding quality, the encoder spends up to $13.8\%$ more time per encode, making such architecture unreliable for implementation on \emph{aomenc}.

Taking this into account, a new approach was employed, using the same \emph{butterfly} scheme as \emph{libaom}'s transforms.

%%%%%%%%%%%%%%%%%%%%%%%%%%%%%%%%%%%%%%%%%%%%%%%%%%%%%%%%%%%%%%%%%%%%%%%%%%%%%%
\section{Alternative \emph{Butterfly} Implementation}

%As mentioned previously, \emph{AV1}'s reference \emph{Transform stage} follows the architecture used in many other \emph{DCT} architectures, and expands its use into the \emph{ADST} \cite{wen-hsiungchenFastComputationalAlgorithm1977}. In this work, the authors 

\clearpage
\printbibliography[heading=subbibliography]
\addcontentsline{toc}{section}{References}

%%%%%%%%%%%%%%%%%%%%%%%%%%%%%%%%%%%%%
% Developed Architecture
\cleardoublepage
\chapter{Conclusions and Future Work}

%%%%%%%%%%%%%%%%%%%%%%%%%%%%%%%%%%%%%%%%%%%%%%%%%%%%%%%%%%%%%%%%%%%%%%%%%%%%%%

This work started with the objective of improving the performance of the recently released video coding standard, \emph{AV1}, by optimization of the reference software, \emph{libaom}, and by the development of hardware architectures for the \emph{Transform} stage.

On that note, the reference software was analyzed on various aspects. Firstly the internal functioning of the focused stage was studied, being described the most relevant features, such as the sequence of operations, internal data structures and implemented kernels.

In addition to these, statistical data referring to the encoding choices done in the encoder was acquired. Upon its analysis, some possible characteristics for improvement of the reference software were found. From the number of bits used in the cosine approximations, to the verified number of occurrences of symmetric kernel blocks, there is a high margin for improvement of the encoder's performance.

On the focus of this work, the tackled measure was the reduction of the number of bits used by cosine approximations, as it was verified that these had a low impact on the obtained video quality, while influencing the overall encoder's performance. With the changes performed in the reference software there was a $3\%$ reduction in the encoding time.

The algorithm was then described in hardware, being achieved two different architectures, one of such was implemented and tested on a \emph{Nexys 4} FPGA. Although the design's functioning was validated, the  performance obtained with the tested kit is not adequate for implementation on a real time encoder for the currently desired resolutions.

Accordingly, it can be concluded that the original objectives were partially achieved. \emph{Libaom}'s \emph{Transform} stage was improved with the software changes, and two different hardware implementations were constructed for the \emph{DCT} kernel. However, \emph{AV1} supports two other transformation kernels, being these the \emph{ADST} and \emph{Identity}. 

With that said, the work started in this dissertation can be profoundly extended, in order to obtain an efficient hardware architecture for \emph{AV1}.

As to the developed architectures, the most immediate measure would be the shortening of the internal signals, as well as the input and output coefficients. Although this measure would bring some complexity to the interconnection of the internal blocks, it would reduce the necessary footprint.

However, this measure wouldn't suffice to improve the architecture's performance by a high margin. To do so, an ASIC implementation should be considered, since, in most cases, FPGA designs tend to perform worse than specialized integrated circuits. This happens because, on an FPGA, the design will always be limited by the hardware kit's logic structure, while on ASIC implementations the implemented hardware is optimized for the design, according to a library of logic cells, on a specific technology, e.g., 90nm \cite{FPGAVsASIC2016}.

Considering an encoder application, an architecture based on the first implementation could better benefit of an ASIC implementation, as this allowed for the parallelization of transformations.

However, the future work isn't limited to the \emph{DCT} kernel. Taking a similar workflow, other architectures could be achieved for both the \emph{ADST} and \emph{IDTX} kernels. In addition, other tasks remnant of the \emph{Transform} stage could benefit from hardware implementations, such as the column-row transpositions done between 1D transformations.

\clearpage
\printbibliography[heading=subbibliography]
\addcontentsline{toc}{section}{References}

%%%%%%%%%%%%%%%%%%%%%%%%%%%%%%%%%%%%%
% Developed Architecture
\cleardoublepage
\renewcommand{\thesection}{\Alph{section}}
\begin{appendices}

\section{\emph{aomenc} Configuration Options} \label{app:libaom}
\begin{lstlisting}

Usage: ./aomenc <options> -o dst_filename src_filename 

Options:
            --help                      Show usage options and exit
-c <arg>,   --cfg=<arg>                 Config file to use
-D,         --debug                     Debug mode (makes output deterministic)
-o <arg>,   --output=<arg>              Output filename
            --codec=<arg>               Codec to use
-p <arg>,   --passes=<arg>              Number of passes (1/2)
            --pass=<arg>                Pass to execute (1/2)
            --fpf=<arg>                 First pass statistics file name
            --limit=<arg>               Stop encoding after n input frames
            --skip=<arg>                Skip the first n input frames
            --good                      Use Good Quality Deadline
-q,         --quiet                     Do not print encode progress
-v,         --verbose                   Show encoder parameters
            --psnr                      Show PSNR in status line
            --webm                      Output WebM (default when WebM IO is enabled)
            --ivf                       Output IVF
            --obu                       Output OBU
-P,         --output-partitions         Makes encoder output partitions. Requires IVF output!
            --q-hist=<arg>              Show quantizer histogram (n-buckets)
            --rate-hist=<arg>           Show rate histogram (n-buckets)
            --disable-warnings          Disable warnings about potentially incorrect encode settings.
-y,         --disable-warning-prompt    Display warnings, but do not prompt user to continue.
            --test-decode=<arg>         Test encode/decode mismatch
                                        off, fatal, warn

Encoder Global Options:
            --yv12                      Input file is YV12 
            --i420                      Input file is I420 (default)
            --i422                      Input file is I422
            --i444                      Input file is I444
-u <arg>,   --usage=<arg>               Usage profile number to use
-t <arg>,   --threads=<arg>             Max number of threads to use
            --profile=<arg>             Bitstream profile number to use
-w <arg>,   --width=<arg>               Frame width
-h <arg>,   --height=<arg>              Frame height
            --forced_max_frame_width    Maximum frame width value to force
            --forced_max_frame_height   Maximum frame height value to force
            --stereo-mode=<arg>         Stereo 3D video format
                                        mono, left-right, bottom-top, top-bottom, right-left
            --timebase=<arg>            Output timestamp precision (fractional seconds)
            --fps=<arg>                 Stream frame rate (rate/scale)
            --global-error-resilient=<  Enable global error resiliency features
-b <arg>,   --bit-depth=<arg>           Bit depth for codec (8 for version <=1, 10 or 12 for version 2)
                                        8, 10, 12
            --lag-in-frames=<arg>       Max number of frames to lag
            --large-scale-tile=<arg>    Large scale tile coding (0: off (default), 1: on)
            --monochrome                Monochrome video (no chroma planes)
            --full-still-picture-hdr    Use full header for still picture

Rate Control Options:
            --drop-frame=<arg>          Temporal resampling threshold (buf %)
            --resize-mode=<arg>         Frame resize mode
            --resize-denominator=<arg>  Frame resize denominator
            --resize-kf-denominator=<a  Frame resize keyframe denominator
            --superres-mode=<arg>       Frame super-resolution mode
            --superres-denominator=<ar  Frame super-resolution denominator
            --superres-kf-denominator=  Frame super-resolution keyframe denominator
            --superres-qthresh=<arg>    Frame super-resolution qindex threshold
            --superres-kf-qthresh=<arg  Frame super-resolution keyframe qindex threshold
            --end-usage=<arg>           Rate control mode
                                        vbr, cbr, cq, q
            --target-bitrate=<arg>      Bitrate (kbps)
            --min-q=<arg>               Minimum (best) quantizer
            --max-q=<arg>               Maximum (worst) quantizer
            --undershoot-pct=<arg>      Datarate undershoot (min) target (%)
            --overshoot-pct=<arg>       Datarate overshoot (max) target (%)
            --buf-sz=<arg>              Client buffer size (ms)
            --buf-initial-sz=<arg>      Client initial buffer size (ms)
            --buf-optimal-sz=<arg>      Client optimal buffer size (ms)

Twopass Rate Control Options:
            --bias-pct=<arg>            CBR/VBR bias (0=CBR, 100=VBR)
            --minsection-pct=<arg>      GOP min bitrate (% of target)
            --maxsection-pct=<arg>      GOP max bitrate (% of target)

Keyframe Placement Options:
            --enable-fwd-kf=<arg>       Enable forward reference keyframes
            --kf-min-dist=<arg>         Minimum keyframe interval (frames)
            --kf-max-dist=<arg>         Maximum keyframe interval (frames)
            --disable-kf                Disable keyframe placement

AV1 Specific Options:
            --cpu-used=<arg>            CPU Used (0..8)
            --dev-sf=<arg>              Dev Speed (0..255)
            --auto-alt-ref=<arg>        Enable automatic alt reference frames
            --sharpness=<arg>           Loop filter sharpness (0..7)
            --static-thresh=<arg>       Motion detection threshold
            --single-tile-decoding=<ar  Single tile decoding (0: off (default), 1: on)
            --tile-columns=<arg>        Number of tile columns to use, log2
            --tile-rows=<arg>           Number of tile rows to use, log2 (set to 0 while threads > 1)
            --arnr-maxframes=<arg>      AltRef max frames (0..15)
            --arnr-strength=<arg>       AltRef filter strength (0..6)
            --tune=<arg>                Distortion metric tuned with
                                        psnr, ssim, cdef-dist, daala-dist
            --cq-level=<arg>            Constant/Constrained Quality level
            --max-intra-rate=<arg>      Max I-frame bitrate (pct)
            --max-inter-rate=<arg>      Max P-frame bitrate (pct)
            --gf-cbr-boost=<arg>        Boost for Golden Frame in CBR mode (pct)
            --lossless=<arg>            Lossless mode (0: false (default), 1: true)
            --enable-cdef=<arg>         Enable the constrained directional enhancement filter (0: false, 1: true (default))
            --enable-restoration=<arg>  Enable the loop restoration filter (0: false, 1: true (default))
            --disable-trellis-quant=<a  Disable trellis optimization of quantized coefficients (0: false (default) 1: true)
            --enable-qm=<arg>           Enable quantisation matrices (0: false (default), 1: true)
            --qm-min=<arg>              Min quant matrix flatness (0..15), default is 8
            --qm-max=<arg>              Max quant matrix flatness (0..15), default is 15
            --enable-dist-8x8=<arg>     Enable dist-8x8 (0: false (default), 1: true)
            --frame-parallel=<arg>      Enable frame parallel decodability features (0: false (default), 1: true)
            --error-resilient=<arg>     Enable error resilient features (0: false (default), 1: true)
            --aq-mode=<arg>             Adaptive quantization mode (0: off (default), 1: variance 2: complexity, 3: cyclic refresh)
            --deltaq-mode=<arg>         Delta qindex mode (0: off (default), 1: deltaq 2: deltaq + deltalf)
            --frame-boost=<arg>         Enable frame periodic boost (0: off (default), 1: on)
            --noise-sensitivity=<arg>   Noise sensitivity (frames to blur)
            --tune-content=<arg>        Tune content type
                                        default, screen
            --cdf-update-mode=<arg>     CDF update mode for entropy coding (0: no CDF update; 1: update CDF on all frames(default); 2: selectively update CDF on some frames
            --color-primaries=<arg>     Color primaries (CICP) of input content:
                                        bt709, unspecified, bt601, bt470m, bt470bg, smpte240, film, bt2020, xyz, smpte431, smpte432, ebu3213
            --transfer-characteristics  Transfer characteristics (CICP) of input content:
                                        unspecified, bt709, bt470m, bt470bg, bt601, smpte240, lin, log100, log100sq10, iec61966, bt1361, srgb, bt2020-10bit, bt2020-12bit, smpte2084, hlg, smpte428
            --matrix-coefficients=<arg  Matrix coefficients (CICP) of input content:
                                        identity, bt709, unspecified, fcc73, bt470bg, bt601, smpte240, ycgco, bt2020ncl, bt2020cl, smpte2085, chromncl, chromcl, ictcp
            --chroma-sample-position=<  The chroma sample position when chroma 4:2:0 is signaled:
                                        unknown, vertical, colocated
            --min-gf-interval=<arg>     min gf/arf frame interval (default 0, indicating in-built behavior)
            --max-gf-interval=<arg>     max gf/arf frame interval (default 0, indicating in-built behavior)
            --sb-size=<arg>             Superblock size to use
                                        dynamic, 64, 128
            --num-tile-groups=<arg>     Maximum number of tile groups, default is 1
            --mtu-size=<arg>            MTU size for a tile group, default is 0 (no MTU targeting), overrides maximum number of tile groups
            --timing-info=<arg>         Signal timing info in the bitstream (model unly works for no hidden frames, no super-res yet):
                                        unspecified, constant, model
            --film-grain-test=<arg>     Film grain test vectors (0: none (default), 1: test-1  2: test-2, ... 16: test-16)
            --film-grain-table=<arg>    Path to file containing film grain parameters
            --enable-ref-frame-mvs=<ar  Enable temporal mv prediction (default is 1)
-b <arg>, --bit-depth=<arg>           Bit depth for codec (8 for version <=1, 10 or 12 for version 2)
                                        8, 10, 12
            --input-bit-depth=<arg>     Bit depth of input
            --sframe-dist=<arg>         S-Frame interval (frames)
            --sframe-mode=<arg>         S-Frame insertion mode (1..2)
            --annexb=<arg>              Save as Annex-B

Stream timebase (--timebase):
The desired precision of timestamps in the output, expressed
in fractional seconds. Default is 1/1000.

Included encoders:

av1    - AOMedia Project AV1 Encoder v0.1.0 (default)

    Use --codec to switch to a non-default encoder.
\end{lstlisting}

%%%%%%%%%%%%%%%%%%%%%%%%%%%%%%%%%%%%%%%%%%%%%%%%%%%%%%%%%%%%%%%%%%%%%%%%%%%
\section{DCT8\_1 VHDL Description} \label{app:dct81}
\begin{lstlisting}[style=vhdl]
-- DCT8 Implementation for inetgration with aggregated architecture

library IEEE;
use IEEE.STD_LOGIC_1164.ALL;
use IEEE.NUMERIC_STD.ALL;
use IEEE.STD_LOGIC_UNSIGNED.ALL;

entity DCT8_1_I is
port(   -- Data Inputs
    dataIn0     : in    integer;
    dataIn1     : in    integer;
    dataIn2     : in    integer;
    dataIn3     : in    integer;
    dataIn4     : in    integer;
    dataIn5     : in    integer;
    dataIn6     : in    integer;
    dataIn7     : in    integer;
    -- Control Inputs
    res         : in    std_logic;
    en          : in    std_logic;
    clk         : in    std_logic;
    -- Data Outputs
    dataOut0    : out    integer;
    dataOut1    : out    integer;
    dataOut2    : out    integer;
    dataOut3    : out    integer;
    dataOut4    : out    integer;
    dataOut5    : out    integer;
    dataOut6    : out    integer;
    dataOut7    : out    integer;
    -- Control Outputs
    validOut    : out   std_logic
);
end DCT8_1_I;

architecture Behavioral of DCT8_1_I is
begin

stage1:     process(clk, res, en)
        begin
            if(rising_edge(clk)) then
                if(res = '1') then
                    dataOut0 <= 0;
                    dataOut1 <= 0;
                    dataOut2 <= 0;
                    dataOut3 <= 0;
                    dataOut4 <= 0;
                    dataOut5 <= 0;
                    dataOut6 <= 0;
                    dataOut7 <= 0;
                    validOut <= '0';
                elsif(en = '1') then
                    dataOut0 <= dataIn0 + dataIn7;
                    dataOut1 <= dataIn1 + dataIn6;
                    dataOut2 <= dataIn2 + dataIn5;
                    dataOut3 <= dataIn3 + dataIn4;
                    dataOut4 <= dataIn3 - dataIn4;
                    dataOut5 <= dataIn2 - dataIn5;
                    dataOut6 <= dataIn1 - dataIn6;
                    dataOut7 <= dataIn0 - dataIn7;
                    validOut <= '1';
                end if;
            end if;
        end process;
end Behavioral;
\end{lstlisting}

%%%%%%%%%%%%%%%%%%%%%%%%%%%%%%%%%%%%%%%%%%%%%%%%%%%%%%%%%%%%%%%%%%%%%%%%%%%
\section{DCT8\_2 VHDL Description} \label{app:dct82}
\begin{lstlisting}[style=vhdl]
-- DCT8 Stage 2 Implementation for integration with aggregated architecture

library IEEE;
use IEEE.STD_LOGIC_1164.ALL;
use IEEE.NUMERIC_STD.ALL;
use IEEE.STD_LOGIC_UNSIGNED.ALL;

entity DCT8_2_I is
port(   -- Data Inputs
    dataIn4     : in    integer;
    dataIn5     : in    integer;
    dataIn6     : in    integer;
    dataIn7     : in    integer;
    -- Control Inputs
    res         : in    std_logic;
    en          : in    std_logic;
    clk         : in    std_logic;
    -- Data Outputs
    dataOut4    : out    integer;
    dataOut5    : out    integer;
    dataOut6    : out    integer;
    dataOut7    : out    integer;
    -- Control Outputs
    validOut    : out   std_logic
);
end DCT8_2_I;

architecture Behavioral of DCT8_2_I is
signal s_stg2M5, s_stg2M6       :   integer := 0;
signal s_stg2A5, s_stg2A6       :   integer := 0;
signal s_stg2D5, s_stg2D6       :   integer := 0;
signal s_stg34, s_stg35, s_stg36, s_stg37       :   integer := 0;
signal s_stg4M41, s_stg4M42, s_stg4M51, s_stg4M52, s_stg4M61, s_stg4M62, s_stg4M71, s_stg4M72       :   integer := 0;
signal s_stg4A4, s_stg4A5, s_stg4A6, s_stg4A7       :   integer := 0;
signal s_stage2MEn, s_stage2AEn, s_stage2DEn, s_stage3En, s_stage4MEn, s_stage4AEn, s_valOut       :   std_logic := '0';
begin

stage2M:    process(clk, res, en)
        begin
            if(rising_edge(clk)) then
                if(res = '1') then
                    s_stg2M5 <= 0;
                    s_stg2M6 <= 0;
                    s_stage2AEn <= '0';
                elsif(en = '1') then
                    s_stg2M5 <= dataIn5*185;
                    s_stg2M6 <= dataIn6*185;
                    s_stage2AEn <= '1';
                end if;
            end if;
        end process;

stage2A:    process(clk, res, s_stage2AEn)
        begin
            if(rising_edge(clk)) then
                if(res = '1') then
                    s_stg2A5 <= 0;
                    s_stg2A6 <= 0;
                    s_stage2DEn <= '0';
                elsif(s_stage2AEn = '1') then
                    s_stg2A5 <= s_stg2M6 - s_stg2M5;
                    s_stg2A6 <= s_stg2M6 + s_stg2M5;
                    s_stage2DEn <= '1';
                end if;
            end if;
        end process;

stage2D:    process(clk, res, s_stage2DEn)
        begin
            if(rising_edge(clk)) then
                if(res = '1') then
                    s_stg2D5 <= 0;
                    s_stg2D6 <= 0;
                    s_stage3En <= '0';
                elsif(s_stage2DEn = '1') then
                    s_stg2D5 <= to_integer(shift_right(to_signed(s_stg2A5,32),8));
                    s_stg2D6 <= to_integer(shift_right(to_signed(s_stg2A6,32),8));
                    s_stage3En <= '1';
                end if;
            end if;
        end process;                

stage3:     process(clk, res, s_stage3En)
        begin
            if(rising_edge(clk)) then
                if(res = '1') then
                    s_stg34 <= 0;
                    s_stg35 <= 0;
                    s_stg36 <= 0;
                    s_stg37 <= 0;
                    s_stage4MEn <= '0';
                elsif(s_stage3En = '1') then
                    s_stg34 <= dataIn4 + s_stg2D5;
                    s_stg35 <= dataIn4 - s_stg2D5;
                    s_stg36 <= dataIn7 - s_stg2D6;
                    s_stg37 <= dataIn7 + s_stg2D6;
                    s_stage4MEn <= '1';
                end if;
            end if;
        end process;

stage4M:    process(clk, res, s_stage4MEn)
        begin
            if(rising_edge(clk)) then
                if(res = '1') then
                    s_stg4M41 <= 0;
                    s_stg4M42 <= 0;
                    s_stg4M51 <= 0;
                    s_stg4M52 <= 0;
                    s_stg4M61 <= 0;
                    s_stg4M62 <= 0;
                    s_stg4M71 <= 0;
                    s_stg4M72 <= 0;
                    s_stage4AEn <= '0';
                elsif(s_stage4MEn = '1') then
                    s_stg4M41 <= s_stg34*56;
                    s_stg4M42 <= s_stg34*252;
                    s_stg4M51 <= s_stg35*147;
                    s_stg4M52 <= s_stg35*216;
                    s_stg4M61 <= s_stg36*147;
                    s_stg4M62 <= s_stg36*216;
                    s_stg4M71 <= s_stg37*56;
                    s_stg4M72 <= s_stg37*252;
                    s_stage4AEn <= '1';
                end if;
            end if;
        end process;

stage4A:    process(clk, res, s_stage4AEn)
        begin
            if(rising_edge(clk)) then
                if(res = '1') then
                    s_stg4A4 <= 0;
                    s_stg4A5 <= 0;
                    s_stg4A6 <= 0;
                    s_stg4A7 <= 0;
                    s_valOut <= '0';
                elsif(s_stage4AEn = '1') then
                    s_stg4A4 <= s_stg4M41 + s_stg4M72;
                    s_stg4A5 <= s_stg4M52 + s_stg4M61;
                    s_stg4A6 <= s_stg4M62 - s_stg4M51;
                    s_stg4A7 <= s_stg4M71 - s_stg4M42;
                    s_valOut <= '1';
                end if;
            end if;
        end process;
        
outReg:     process(clk, res, s_valOut)
        begin
            if(rising_edge(clk)) then
                if(res = '1') then
                    dataOut5 <= 0;
                    dataOut6 <= 0;
                    dataOut7 <= 0;
                    validOut <= '0';
                elsif(s_valOut = '1') then
                    dataOut4 <= to_integer(shift_right(to_signed(s_stg4A4,32),8));
                    dataOut5 <= to_integer(shift_right(to_signed(s_stg4A5,32),8));
                    dataOut6 <= to_integer(shift_right(to_signed(s_stg4A6,32),8));
                    dataOut7 <= to_integer(shift_right(to_signed(s_stg4A7,32),8));
                    validOut <= '1';
                end if;
            end if;
        end process;
end Behavioral;
\end{lstlisting}
\end{appendices}


%%%%%%%%%%%%%%%%%%%%%%%%%%%%%%%%%%%%%
% Bibliografia
%\cleardoublepage
%  \nocite{*}
%  \printbibliography
%\cleardoublepage

\end{document}
