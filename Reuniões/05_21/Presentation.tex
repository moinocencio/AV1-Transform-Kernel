\documentclass{beamer}
%
% Choose how your presentation looks.
%
% For more themes, color themes and font themes, see:
% http://deic.uab.es/~iblanes/beamer_gallery/index_by_theme.html
%
\mode<presentation>
{
\usetheme{Madrid}      % Madrid, Montpellier, Pittsburgh, Rochester, boxes
\usecolortheme{seagull} 	% beaver, crane, dolphin, dove, lily, orchid, seagull, seahorse
\usefonttheme{default}  % or try serif, structurebold, ...
\setbeamertemplate{navigation symbols}{}
\setbeamertemplate{caption}[numbered]
}

\usepackage[portuguese]{babel}
\usepackage[utf8x]{inputenc}
\usepackage{multirow}
\usepackage{ragged2e}
\usepackage{textpos}
\usepackage[export]{adjustbox}
\justifying
\usepackage{pifont}
\newcommand{\xmark}{\ding{55}}%
\usepackage{tikz}
\usetikzlibrary{arrows,positioning} 
\usetikzlibrary{matrix}
\tikzset{
    %Define standard arrow tip
    >=stealth',
    %Define style for boxes
    reBox/.style={
           rectangle,
           rounded corners,
           draw=black, very thick,
           text width=6.5em,
           minimum height=2em,
           text centered},
    rBox/.style={
           rectangle,           
           draw=black, very thick,
           minimum height=2em,
           text centered},
    % Define arrow style
    pil/.style={
           ->,
           thick,
           shorten <=2pt,
           shorten >=2pt,}
}
\def\checkmark{\tikz\fill[scale=0.4](0,.35) -- (.25,0) -- (1,.7) -- (.25,.15) -- cycle;} 


\setlength{\parindent}{0.5cm}
%%%%%%%%%%%%%%%%%%%%%
% Títulos e etc
\title[Apresentação Semanal]{Tese de Mestrado}
\subtitle{Apresentação Semanal}
\author[M. Inocêncio]{Miguel Inocêncio}
\institute[UA]	{Universidade de Aveiro\\ 
				Instituto de Telecomunicações}
\date{21/05/2019}
\titlegraphic{\includegraphics[height=1.5cm]{../ua.jpg}\includegraphics[height=1.5cm]{../IT.png}}

\begin{document}

%%%%%%%%%%%%%%%%%%%%%
% Página Inicial
\begin{frame}
	\titlepage
\end{frame}

%%%%%%%%%%%%%%%%%%%%%
% Table of Contents
\begin{frame}{Conteúdos}
	\tableofcontents
\end{frame}

%%%%%%%%%%%%%%%%%%%%%%
% Libaom
\section{Libaom - Software de Referência}
\begin{frame}
	\frametitle{Libaom - Ponto de situação na última reunião}

	\begin{itemize}
		\item Encoding, Decoding e visualização realizadas com sucesso
		\item Dificuldade no estudo do código fonte
		\item Realização do setup para debug com Visual Studio 		
	\end{itemize}

\end{frame}

\begin{frame}
	\frametitle{Libaom - Desenvolvimento do estudo do código fonte}

	%Tendo em conta a reBoxexidade do código e a falta de experiência com um projeto desta dimensão, a melhor ferramenta disponível seria fazer um debug sólido linha-a-linha e verificar o procedimento do código.

       \centering
       \begin{tikzpicture}
              \node (dummy) {};
              \coordinate[below=of dummy] (c);
              \node[reBox, left=of c] (complexidade) {Complexidade do Código};
              \node[reBox, right=of c] (experiencia) {Falta de experiencia com projetos desta dimensão};
              \node[reBox, below=2cm of c] (debug) {Debug linha-a-linha com VSCode};

              \draw [pil, bend right=45] (complexidade) -- (debug);
              \draw [pil, bend left=45] (experiencia) -- (debug);
       \end{tikzpicture}
\end{frame}

\begin{frame}
       \frametitle{Libaom - Desenvolvimento do debug}

       O debug e procedente análise do código foi feita realizando o encode de uma sequência de frames linha a linha.

       Deste modo foi possível verificar as condições de execução do código, e os diferentes módulos usados, não sendo esta análise apresentada nesta reunião.
       
       \begin{itemize}
              \item \textbf{Organização de fluxograma de código?} 
       \end{itemize}
\end{frame}

\begin{frame}
       \frametitle{Libaom - Cálculo da Transformada}

       Como mencionado numa reunião anterior, o AV1 suporta mais tipos de transformadas que os seus antecedentes, bem como uma maior diversidade de tamanhos, com a adição de formatos retangulares.

       No \textit{libaom}, este aspeto é realizado efetuando a transformada independentemente entre as colunas e as linhas, a 1 dimensão, permitindo assim que o tamanho das duas dimensões seja diferente, efetuando assim uma transformada em bloco retangular.

       \begin{tikzpicture} [ampersand replacement=\&,
                            nodes in empty cells,
                            nodes={minimum width=0.1cm, 
                            minimum height=0.7cm},
                            row sep=-\pgflinewidth, 
                            column sep=-\pgflinewidth]

              border/.style={draw}
            
              \matrix(vector1)     [matrix of nodes,
                                   nodes in empty cells,
                                   row 1/.style={nodes={draw, minimum width=0.1cm,minimum height=1cm},anchor=center}]
                                   {
                                   $a_{0,0}$ \& $a_{0,1}$ \& \dots \& $a_{0,N-1}$\\
                                   };

              \node  [rBox, 
                     right=0.5cm of vector1,
                     minimum height=1.2cm,
                     minimum width=1.2cm]                      
                     (T1) {$T$};
              
              \draw [pil] (vector1) -- (T1);
              
              \matrix(vector2)     [right=of T1,
                                   matrix of nodes,
                                   row 1/.style={nodes={draw, minimum width=1.4cm,minimum height=0.7cm}},
                                   row 2/.style={nodes={draw, minimum width=1.4cm,minimum height=1cm}},
                                   row 3/.style={nodes={draw, minimum width=1.4cm,minimum height=1cm}},
                                   row 4/.style={nodes={draw, minimum width=1.4cm,minimum height=1cm}}]
                                   {
                                   $a_{0,0}$ \\
                                   $a_{1,0}$ \\
                                   $...$ \\
                                   $a_{M-1,0}$\\
                                   };

              \node  [rBox, 
                     right=0.5cm of vector2,
                     minimum height=1.5cm,
                     minimum width=1.5cm]                      
                     (T2) {$T$};
              
              \draw [pil] (vector2) -- (T2);
        \end{tikzpicture}
\end{frame}

\begin{frame}
       \frametitle{Libaom - Transformadas Implementadas}

       Da análise que do código, as transformadas implementadas são:

       \begin{table}%[ampersand replacement=\&]
              \centering
              \begin{tabular}{c|r|r|r|r|r}
                     \textbf{Transformada} & $1 \times 4$ & $1 \times 8$ & $1 \times 16$ & $1 \times 32$ & $1 \times 64$ \\ \hline
                     DCT & \checkmark & \checkmark & \checkmark & \checkmark & \checkmark \\ \hline
                     ADST & \checkmark & \checkmark & \checkmark & \xmark & \xmark \\ \hline
                     Identidade & \checkmark & \checkmark & \checkmark & \checkmark & \xmark \\                      
              \end{tabular}
       \end{table}
\end{frame}

\begin{frame}
       \frametitle{Libaom - Próximos passos}

       Tendo em conta esta análise inicial, apresento uma proposta para os próximos passos:

       \begin{itemize}
              \item Fazer o encode de diferentes sequências de vídeos, de modo a analisar quais as transformadas utilizadas com maior frequência
              \item Diagramas de blocos para as transformadas
              \item Estudo teórico das diferentes transformadas e comparação com os modelos implementados em software
       \end{itemize}

\end{frame}

%%%%%%%%%%%%%%%%%%%%%%
% Cadence
\section{Cadence}
\begin{frame}
       \frametitle{Obtenção de ferramentas do Cadence}

       A obtenção do \textit{IP Package} apresenta um custo elevado para a obtenção de vários programas, muitos dos quais não seriam utilizados.

       O Jones sugeriu se seria possível obter apenas o programa \textit{Genus}, pois segundo ele, seria o programa que teria maior utilidade.

       A decisão quanto à obtenção dos programas poderia ser tomada consoante esta disponibilidade.

       Quanto à disponibilidade de utilizar as ferramentas deles, não foi dada nenhuma informação adicional, para além da informação que já tinha sido dada anteriormente (utilização das ferramentas, aquando da minha deslocação a Pelotas, ou utilização remota).
\end{frame}

%%%%%%%%%%%%%%%%%%%%%%
% JVET
\section{VVC}
\begin{frame}
       \frametitle{Análise VVC-AV1}

       O VVC é o mais recente padrão do JVET, ainda sem finalização do bitstream anunciada.

       Apesar da sua performance exemplar em comparação com o HEVC e até em relação ao AV1, há uma incógnita quanto à tendência da indústria na utilizaçaão das diferentes opções, devido à opção da AOM para tornar os seus codecs \textit{royalty free}.
\end{frame}


\end{document}	