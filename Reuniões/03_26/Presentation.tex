\documentclass{beamer}
%
% Choose how your presentation looks.
%
% For more themes, color themes and font themes, see:
% http://deic.uab.es/~iblanes/beamer_gallery/index_by_theme.html
%
\mode<presentation>
{
\usetheme{default}      % or try Darmstadt, Madrid, Warsaw, ...
\usecolortheme{beaver} % or try albatross, beaver, crane, ...
\usefonttheme{default}  % or try serif, structurebold, ...
\setbeamertemplate{navigation symbols}{}
\setbeamertemplate{caption}[numbered]
}

\usepackage[portuguese]{babel}
\usepackage[utf8x]{inputenc}
\usepackage{multirow}

\setlength{\parindent}{0.5cm}
%%%%%%%%%%%%%%%%%%%%%
% Títulos e etc
\title[Apresentação Semanal]{Apresentação Semanal}
\author{Miguel Inocêncio}
\institute{Universidade de Aveiro}
\date{26/03/2019}

\begin{document}

%%%%%%%%%%%%%%%%%%%%%
% Página Inicial
\begin{frame}
	\titlepage
\end{frame}

%%%%%%%%%%%%%%%%%%%%%
% Table of Contents
\begin{frame}{Conteúdos}
	\tableofcontents
\end{frame}

%%%%%%%%%%%%%%%%%%%%%%
% Introduções Gerais
\section{HEVC - Introdução Geral}
\begin{frame}{HEVC}
	\begin{itemize}
		\item Projeto lançado em 2010 pela ITU-T Video Coding Experts Group (VCEG) e ISSO/IEC Moving Picture Experts Group (MPEG)
		\item Originou uma nova organização: Joint Collaborative Team on Video Coding (JCT-VC)
		\item Primeira versão lançada em 2013
		\item Sucessor do H.264/AVC
	\end{itemize}

	\begin{figure}
		\includegraphics[height=0.4\textheight]{holders.png}
		\caption{\label{fig:hevc_holders}Detentores de patentes do HEVC}
	\end{figure}
\end{frame}

\section{AV1 - Introdução Geral}
\begin{frame}{AV1}
	\begin{itemize}
		\item Formato de compressão Open Source e sem royalties
		\item Finalizada a primeira versão em 2018, pela Alliance for Open Media (AOMedia)
		\item Sucessor do VP9 (formato da Google, usado no Youtube)
		\item Desenvolvido para aplicações de streaming
	\end{itemize}

	\begin{figure}
		\includegraphics[height=0.5\textheight]{aom.png}
		\caption{\label{fig:AOM}Alliance for Open Media}
	\end{figure}
\end{frame}

%%%%%%%%%%%%%%%%%%%%%
% Comparação Técnica HEVC AV1
\section{Comparação técnica entre HEVC vs AV1}

\begin{frame}{Comparação técnica entre HEVC vs AV1}
	\begin{itemize}
		\item Apresentação geral de ambos os formatos de codificação
		\item Comparação de aspetos técnicos
	\end{itemize}
\end{frame}

\subsection{High Level Syntax}
\begin{frame}{High Level Syntax}
	\begin{table}
		\centering
		\begin{tabular}{l|l|l}
			\textbf{Característica} 					& \textbf{HEVC} 	& \textbf{AV1} \\\hline
			\textbf{Perfis} 									& 14 							& 3 \\
			\textbf{Níveis} 									& 13 							& 12 \\
			\multirow{2}{*}{\textbf{Layers}} 	& Slices 					& Frames divididas \\
																				& independentes 	& por Tiles \\
		\end{tabular}
		\caption{\label{tab:high-level}Noções gerais de ambos os codecs}
	\end{table}
\end{frame}

\subsection{Partições}
\begin{frame}{Partições}
	Tamanho máximo \textit{Coding Tree Units}'s no HEVC e \textbf{superblock}'s no AV1, bem como todos os tamanhos usados nas diferentes fases do processo.
	\begin{table}
		\centering
		\begin{tabular}{l|l|l}
			\textbf{Característica} 					& \textbf{HEVC} 	& \textbf{AV1} \\\hline
			\textbf{Nº de tamanhos} 						& 24 							& 42 \\
			\textbf{Tamanho máximo} 					& 64x64 					& 128x128
		\end{tabular}
		\caption{\label{tab:partitioning}Partitioning}
	\end{table}
\end{frame}

\subsection{Intra-Prediction}
\begin{frame}{Intra-Prediction}
	Ambos os formatos usam processos semelhantes, embora o HEVC apenas melhore as tecnologias implementadas no AVC. Neste aspeto, o AV1 adiciona funcionalidades inexistentes no VP9.
	\begin{table}
		\centering
		\begin{tabular}{l|l|l}
			\textbf{Característica} 									& \textbf{HEVC} 	& \textbf{AV1} \\\hline
			\textbf{Nº de modos angulares} 						& 33 							& 56 \\
			\textbf{Nº modos não angulares} 						& 2	 							& 6 \\
			\multirow{2}{*}{\textbf{Outras adições}}	& $\varnothing$		& 5 modos recursivos \\
			 																					&									& 1 \textit{Chroma from Luma} \\
		\end{tabular}
		\caption{\label{tab:intra}Intra-Prediction}
	\end{table}
\end{frame}

\subsection{Inter-Prediction}
\begin{frame}{Inter-Prediction}
	Novamente, ambos os processos seguem abordagens semelhantes. Contudo, o HEVC é mais exigente em termos de memória, enquanto que o AV1 peca pela exigência em termos de complexidade.
	\begin{table}
		\centering
		\begin{tabular}{l|l|l}
			\textbf{Característica} 									& \textbf{HEVC} 	& \textbf{AV1} \\\hline
			\textbf{Nº de modos de predição} 					& 2 							& 4 \\
			\textbf{Nº frames de referência} 					& 8 de 16	 				& 7 de 8 \\
			\multirow{3}{*}{\textbf{Outras adições}}	& $\frac{1}{8} pel$		& Global Motion \\
			 																					&									& 5 filtros de sub-pel \\
																								&									& independentes \\
		\end{tabular}
		\caption{\label{tab:inter}Inter-Prediction}
	\end{table}
\end{frame}

\subsection{Transformadas}
\begin{frame}{Transformadas}
	Ambos os formatos manteram as técnicas dos seus predecessores, inovando nos tamanhos dos blocos. Quer isto dizer que o AV1 apresenta um grau de liberdade bastante superior ao HEVC.
	\begin{table}
		\centering
		\begin{tabular}{l|l|l}
			\textbf{Característica} 													& \textbf{HEVC} 	& \textbf{AV1} \\\hline
			\multirow{3}{*}{\textbf{Tipos de transformadas}} 	& DCT e DST 			& DCT, ADST, \\
																												&									& Flip ADS e \\
																												&									& Identidade \\
			\textbf{Tamanho máximo do bloco} 									& 32x32		 				& 64x64 \\
			\multirow{2}{*}{\textbf{Outras adições}}					& $\varnothing$		& Blocos Retangulares \\
			 																									&									& Blocos Recursivos \\
		\end{tabular}
		\caption{\label{tab:transforms}Transforms}
	\end{table}
\end{frame}

\subsection{Quantização}
\begin{frame}{Quantização}
	Nenhum dos dois formatos sofreu grandes alterações em relação ao análogo anterior. A quantização é feita através de matrizes fixas, que é escolhida a partir de um parâmetro calculado (QP).
	\begin{table}
		\centering
		\begin{tabular}{l|l|l}
			\textbf{Característica} 													& \textbf{HEVC} 	& \textbf{AV1} \\\hline
			\textbf{Nº de parametros para QP}									& 2					 			& 6 \\
			\textbf{Outras adições}														& $\varnothing$		& Offset para superblocos \\
		\end{tabular}
		\caption{\label{tab:quantization}Quantization}
	\end{table}
\end{frame}

\subsection{Codificação de Entropia}
\begin{frame}{Codificação de Entropia}
	Neste ramo, o HEVC retirou um dos modos de codificação, mantendo apenas o CABAC. No caso do AV1, manteve-se a codificação aritmética do VP9, com o aumento do alfabeto.
	\begin{table}
		\centering
		\begin{tabular}{l|l|l}
			\textbf{Característica} 													& \textbf{HEVC} 	& \textbf{AV1} \\\hline
			\multirow{2}{*}{\textbf{Codificação}}							& CABAC			 			& Multi-symbol arithmetic \\
																												&									& com alfabeto até 16 \\
			\textbf{Atualização do alfabeto}									& a cada frame		& a cada símbolo \\
		\end{tabular}
		\caption{\label{tab:entropy}Entropy Coding}
	\end{table}
\end{frame}

\subsection{Filtragem}
\begin{frame}{Filtragem}
	Ambos os formatos inovaram neste ramo, adicionando filtros opcionais, assim como formalizando a utilização de filtros opcionais nos formatos anteriores.
	\begin{table}
		\centering
		\begin{tabular}{l|l|l}
			\textbf{Característica} 													& \textbf{HEVC} 	& \textbf{AV1} \\\hline
			\textbf{De-blocking}															& Sim			 				& Sim \\
			\multirow{3}{*}{\textbf{Outros Filtros}}					&	\multirow{2}{*}{Sample Adaptive Offset}	& Constrained Directional \\
			 																									&													&  Enhancement Filter\\
																												&													&	Loop Filter \\
		\end{tabular}
		\caption{\label{tab:filtering}Filtering}
	\end{table}
\end{frame}

%%%%%%%%%%%%%%%%%%%%%%%%%%%%
% Comparação de Performance
\section{Análise de Performance}
\begin{frame}{Análise de Performance}
	A performance de ambos os encoders foi avaliada em dois aspetos: qualidade de codificação e tempo de enconding.

	Este último parâmetro está altamente dependente do hardware no qual é implementado, nomeadamente devido à grande maioria das placas gráficas lançadas desde 2016 já possuírem aceleradores de hardware dedicado para encoding/decoding de HEVC. Além disto, também a maturidade dos encoders em software para HEVC e correspondente optimização dos seus processos leva ao aumento da sua vantagem em relação ao AV1.

	Quanto à qualidade objetiva e subjetiva dos formatos, também aqui existe alguma liberdade de resultados, devido aos diferentes perfis a utilizar.
\end{frame}

\begin{frame}{Análise de Performance}
	A complexidade adicional do AV1 é recompensada, devido à qualidade adicional obtida, quando comparada com o HEVC. Contudo, torna-se difícil concluir com um número final, devido à disparidade de resultados encontrada, já que alguns testes mostram acréscimos de 2% em PSNR, enquanto outros mostram 40% de melhorias.

	Quanto ao tempo de codificação, os resultados mais recentes (Julho de 2018) mostram resultados pouco animadores, apesar de terem sido feitas melhorias aos encoders de software posteriormente.

	\begin{table}
		\centering
		\begin{tabular}{l|l|l}
			\textbf{Codificador}		 													& \textbf{Tempo de Encoding (s)} 	& \textbf{x Tempo Real} \\\hline
			\textbf{x265}																			& 289			 												& 58 \\
			\textbf{libaom}																		&	226 080													& 45 216 \\
		\end{tabular}
		\caption{\label{tab:time}Tempo de codificação de clip de 5s}
	\end{table}


\end{frame}

\end{document}
