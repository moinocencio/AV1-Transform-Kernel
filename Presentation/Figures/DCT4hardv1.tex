\begin{tikzpicture}[%
    >={Triangle[length=6pt,angle'=28]},
    start chain=going below ,    % General flow is left-to-right
    node distance=5mm and 20mm, % Global setup of box spacing
    every join/.style={norm},   % Default linetype for connecting boxes
    scale=0.45, 
    every node/.style={transform shape}
    ]

  \tikzset{
    base/.style={draw, on chain, on grid, align=center, minimum height=3.5ex},
    reg/.style={base, rectangle, minimum width=2.5em, minimum height=2em,fill=black!15},
    proc/.style={base, rectangle, minimum width=4em, minimum height=8em, fill=black!15},
    inout/.style={base,trapezium,trapezium left angle=70,trapezium right angle=-70, fill=blue!12},
    term/.style={proc, rounded corners},
    sum/.style={base, circle, inner sep=0pt, radius=0.4cm, fill=black!15},
    % coord node style is used for placing corners of connecting lines
    coord/.style={coordinate, on chain, on grid, node distance=6mm and 40mm},
    % nmark node style is used for coordinate debugging marks% -------------------------------------------------
    % Connector line styles for different parts of the diagram
    norm/.style={ar32, draw},
    free/.style={ar32, draw, green3},
    cong/.style={ar32, draw, red3},
    it/.style={font={\small\itshape}},
    nar/.style={ar32, red!75!black},
    ar32/.style={->, line width=0.8mm},   
    b32/.style={line width=0.8mm},    
    ar1/.style={->, line width=0.4mm}  
  }    

  \begin{scope}[y=-1cm]
    \begin{scope}
      %\draw[step=1cm, very thin, lightgray] (0,0) grid (14,21);
      
      %% Inputs
      \draw [black,very thick] (1,0) circle (1mm) node [right, rotate=90, xshift=1mm] {\texttt{Clock}};
        %\draw [ar32] (1,0.1) |- (5,19.5-1);
        \draw [ar1] (1,0.1) |- (5,15.5-1);
        \draw [ar1] (1,0.1) |- (5,11.5-1);
        \draw [ar1] (1,0.1) |- (5,7.5-1);
        \draw [ar1] (1,0.1) |- (5,3.5-1);
      \draw [black,very thick] (2,0) circle (1mm) node [right, rotate=90, xshift=1mm] {\texttt{Reset}};
        %\draw [ar32] (2,0.1) |- (5,19.5-1.5);
        \draw [ar1] (2,0.1) |- (5,15.5-1.5);
        \draw [ar1] (2,0.1) |- (5,11.5-1.5);
        \draw [ar1] (2,0.1) |- (5,7.5-1.5);
        \draw [ar1] (2,0.1) |- (5,3.5-1.5);
      \draw [black,very thick] (3,0) circle (1mm) node [right, rotate=90, xshift=1mm] {\texttt{Enable}};
        \draw [ar1] (3,0.1) |- (5,1.5);
      \foreach \x in {0,...,3}
        \draw [black,very thick] (6+2*\x,0) circle (1mm) node [right, rotate=90, xshift=1mm] (dIn\x) {\texttt{dataIn\x}};      

      %% In Register
      \node (inReg) at (5,2-1) [draw,thick,minimum width=8cm,minimum height=3cm, rounded corners, anchor=north west, fill=black!8, font={\large}] {\textbf{Stage 1: Sum}};
      \node (enInReg) at (5,2.5-1) [right, font={\small}] {\texttt{en}};
      \node (resInReg) at (5,3-1) [right, font={\small}] {\texttt{res}};
      \node (clkInReg) at (5,3.5-1) [right, font={\small}] {\texttt{clk}};
      \node (valoutInReg) at (5,4.5-1) [right, font={\small}] {\texttt{valOut}};
        \draw [ar1] (5,4.5-1) -- ++(-1,0) |- (5,6.5-1);

      %% Stage 1 Sum
      \node (St1) at (5,6-1) [draw,thick,minimum width=8cm,minimum height=3cm, rounded corners, anchor=north west, fill=black!8, font={\large}] {\textbf{Stage 2: Multiplier}};
      \node (enSt1) at (5,6.5-1) [right, font={\small}] {\texttt{en}};
      \node (resSt1) at (5,7-1) [right, font={\small}] {\texttt{res}};
      \node (clkSt1) at (5,7.5-1) [right, font={\small}] {\texttt{clk}};
      \node (valoutSt1) at (5,8.5-1) [right, font={\small}] {\texttt{valOut}};
        \draw [ar1] (5,8.5-1) -- ++(-1,0) |- (5,10.5-1);

      %% Stage 2 Mult
      \node (St2M) at (5,10-1) [draw,thick,minimum width=8cm,minimum height=3cm, rounded corners, anchor=north west, fill=black!8, font={\large}] {\textbf{Stage 2: Sum}};
      \node (enSt2M) at (5,10.5-1) [right, font={\small}] {\texttt{en}};
      \node (resSt2M) at (5,11-1) [right, font={\small}] {\texttt{res}};
      \node (clkSt2M) at (5,11.5-1) [right, font={\small}] {\texttt{clk}};
      \node (valoutSt2M) at (5,12.5-1) [right, font={\small}] {\texttt{valOut}};
        \draw [ar1] (5,12.5-1) -- ++(-1,0) |- (5,14.5-1);
      
      %% Stage 2 Add
      \node (St2A) at (5,14-1) [draw,thick,minimum width=8cm,minimum height=3cm, rounded corners, anchor=north west, fill=black!8, font={\large}] {\textbf{Stage 2: Shift}};
      \node (enSt2A) at (5,14.5-1) [right, font={\small}] {\texttt{en}};
      \node (resSt2A) at (5,15-1) [right, font={\small}] {\texttt{res}};
      \node (clkSt2A) at (5,15.5-1) [right, font={\small}] {\texttt{clk}};
      \node (valoutSt2A) at (5,16.5-1) [right, font={\small}] {\texttt{valOut}}; 
        \draw [ar1] (5,15.5) -| (2,16.5);
        \draw [line width=0.4mm] (2,16.5) -- ++(0,0.4);

      
      %%% Stage 2 Shift
      %\node (St2S) at (5,18-1) [draw,thick,minimum width=8cm,minimum height=3cm, rounded corners, anchor=north west, fill=black!8, font={\large}] {\textbf{Stage 2: Shift}};
      %\node (enSt2S) at (5,18.5-1) [right, font={\small}] {\texttt{en}};
      %\node (resSt2S) at (5,19-1) [right, font={\small}] {\texttt{res}};
      %\node (clkSt2S) at (5,19.5-1) [right, font={\small}] {\texttt{clk}};
      %\node (valoutSt2S) at (5,20.5-1) [right, font={\small}] {\texttt{valOut}};  
      %  \draw [ar32] (5,20.5-1) -| (2,20.5); 
      %  \draw[line width=0.4mm] (2,20.5) -- ++(0, -0.4cm);
      
      %% Outputs
      \draw [black,very thick] (2,17) circle (1mm) node [left, rotate=90, xshift=-1mm] {\texttt{validOut}};
      
      \draw [black,very thick] (6,17) circle (1mm) node [left, rotate=90, xshift=-1mm] (dOut0) {\texttt{dataOut0}};
      \draw [black,very thick] (6+2*1,17) circle (1mm) node [left, rotate=90, xshift=-1mm] (dOut2) {\texttt{dataOut2}};
      \draw [black,very thick] (6+2*2,17) circle (1mm) node [left, rotate=90, xshift=-1mm] (dOut1) {\texttt{dataOut1}};
      \draw [black,very thick] (6+2*3,17) circle (1mm) node [left, rotate=90, xshift=-1mm] (dOut3) {\texttt{dataOut3}};

      \foreach \x in {0,...,3} {
        \draw[ar32] (6+2*\x,-1mm) -- ++(0, -0.9cm);
        \draw[ar32] (6+2*\x,4) -- ++(0, -1cm);
        \draw[ar32] (6+2*\x,8) -- ++(0, -1cm);
        \draw[ar32] (6+2*\x,12) -- ++(0, -1cm);
        \draw[ar32] (6+2*\x,16) -- ++(0, -0.5cm);
        %\draw[ar32] (6+2*\x,20) -- ++(0, -0.5cm);
        \draw[b32] (6+2*\x,16.5) -- ++(0, -0.4cm);
      }
    \end{scope}
    \begin{pgfonlayer}{background}
      % Left-top corner of the background rectangle
      \node (a11) at (0,0.5) {};
      % Right-bottom corner of the background rectanle
      \node (a21) at (13.5,16.5) {};
      % Draw the background
      \path[fill=black!3,rounded corners, draw, very thick]
        (a11) rectangle (a21);
      \node (a31) at (1.5,16.5) {};
      \path (a11) -- (a31) node[midway, rotate=90, yshift=1.3cm, font={\huge}] (mid_right) {\textbf{DCT4}};
    \end{pgfonlayer}
  \end{scope}
\end{tikzpicture}